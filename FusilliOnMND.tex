\chapter{Fusilli on MND}
\label{fusilli_on_mnd}

\section{Introduction}
\begin{itemize}
    \item We've seen the results of fusilli on open data
    \item Now let's do it on MND data
    \item How different is MND to PD and AD? *Look this up* - what might we expect to see?
    \item Hypothesis: multimodal data fusion is useful for predicting survival in MND when deployed at the diagnostic appointment
    \item Motivation - why is this useful? Implications if multimodal data fusion is better, imaging doesn't have to be just used for differential diagnosis and can play a role in prognosis, which is important for clinical trials and patient care.
\end{itemize}

The aim of this work is to assess the added value of brain ROI volumes in MND prognosis prediction.
We hypothesise that the combination of clinical and imaging data will improve the prediction of survival in MND patients.
We will do this by using the fusilli package to compare the performance of different tabular-tabular multimodal data fusion methods.

\section{Data}

ALS Biomarker's study and Opsedale San Raffaele.
We are using clinical data and brain volumes.

%\begin{itemize}
%    \item Where is the data coming from? Milan and the UK
%    \item Tabular-tabular data fusion. Why?
%    \item Sample size is too small for effective deep learning with 3D MRI or 2D slices of MRI
%    \item Instead we'll look at extracted brain volumes
%    \item Why might this be useful anyway? It's more accessible than raw MRI data for use in hospitals and clinics, potentially easier to interpret with explainable AI than deep learning, where the features are extracted automatically
%\end{itemize}

Inclusion criteria:
- Diagnosis of MND
- Non-missing age, sex, date of diagnosis, date of death, date of onset, features of FTD, ALSFRS-R
- MRI within 12 months of diagnosis (either side)
    - Why within 12 months? Trade-off between having the MRI close to the diagnosis and having enough data to work with

Clinical data features:
- Following the same features as the ENCALS model, and working with what we have available
- sex, bulbar onset, signs of FTD, ALSFRSr, ALS subtype, diagnostic delay, age at diagnosis

Milan date of death is death or tracheostomy

\subsection{Clinical Data}
\begin{itemize}
    \item Inclusion criteria
    \item Features
\end{itemize}

\subsection{Imaging Data}
\begin{itemize}
    \item Inclusion criteria: how far away from diagnosis was the MRI? Within 12 months of the diagnosis
    \item Segmentation using SynthSeg - what is synthseg?
    \item Features
\end{itemize}

\section{Methods}
\begin{itemize}
    \item What are we predicting? Long vs short survival split on the median
    \item What methods are we using? All the tabular-tabular fusion methods available in fusilli, plus a baseline of just using the clinical data or just using the imaging data
    \item K-fold cross validation
    \item To improve the stability of the results, we retrained and reevaluated the models until the mean of the performance of the repetitions converged to *include percentage here*
    \item Metrics: AUROC and accuracy
\end{itemize}

\section{Results}
Model comparison figure.

\section{Discussion}
\subsection{What does it mean??}
Interpreting the results.

\subsection{Limitations}
\begin{itemize}
    \item Limitations on sample size
    \item Predictive task of classification rather than regression: what if we used a regression task instead? Would that be more useful? It's a harder task so may require more data
    \item Limitation on using extracted brain volumes rather than raw MRI: what if the regions we've chosen aren't the most important ones? Subcortical regions have shown to have a role in MND, but we haven't included them here. *Look this up - the thalamus stuff*
\end{itemize}

\section{Conclusion}
First look at multimodal data fusion in MND. What does it mean? What are the implications? What are the next steps?
\begin{itemize}
    \item If imaging + clinical is useful
    \begin{itemize}
        \item Let's add modalities
        \item Let's mix up the imaging preprocessing: DTI? Sub-cortical segmentation?
    \end{itemize}
    \item If imaging + clinical isn't useful
    \begin{itemize}
        \item Let's swap out the imaging for other modalities
        \item Let's try different machine learning models
        \item Let's mix up the imaging preprocessing: DTI? Sub-cortical segmentation?
    \end{itemize}
\end{itemize}
