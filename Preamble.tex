% I may change the way this is done in a future version, 
%  but given that some people needed it, if you need a different degree title 
%  (e.g. Master of Science, Master in Science, Master of Arts, etc)
%  uncomment the following 3 lines and set as appropriate (this *has* to be before \maketitle)
% \makeatletter
% \renewcommand {\@degree@string} {Master of Things}
% \makeatother

\title{Multimodal Data Fusion for Motor Neuron Disease Prognosis Prediction}
\author{Florence J Townend}
\department{Centre for Medical Image Computing and Institute of Health Informatics}

\maketitle
\makedeclaration

\begin{abstract} % 300 word limit

    Motor neuron disease (MND) is a rare neurodegenerative disease with no cure and clinically heterogeneous progression and presentation.
    Accurate prognosis prediction is crucial for patient care and clinical trial planning, predominantly relying on clinical data. 
    However, incorporating neuroimaging data could enhance prognostic accuracy.
    This thesis explores the use of multimodal data for prognosis prediction in MND, aiming to improve the accuracy of survival prediction by combining clinical and neuroimaging data through machine learning techniques, known as multimodal data fusion.

    We investigated how multimodal data affects the outcomes of a simple survival analysis technique, the Cox proportional hazards model, revealing significant predictors of MND survival across both clinical and neuroimaging-derived variables.
    We then explored how more complex machine learning models can be used to predict survival in MND, aiming to capture the complex relationships between the different data modalities through deep learning.
    Since the architectures of multimodal deep learning models vary significantly, we developed a Python library, Fusilli, a tool for comparing different multimodal architectures.
    We used Fusilli to compare ten multimodal and unimodal models for predicting survival over 24 months.
    Model architecture significantly influenced performance, with the best model achieving an area under the receiver operating characteristic curve of 0.81.

    Future efforts will focus on enhancing experiments with larger sample sizes and refined patient characterisation, exploring diverse imaging-derived features' impact on model performance, integrating additional data modalities, and evaluating clinical applicability through various prognostic definitions.

\end{abstract}



% \begin{acknowledgements}
% Acknowledgements!
% \end{acknowledgements}

\setcounter{tocdepth}{2} 
% Setting this higher means you get contents entries for
%  more minor section headers.

\tableofcontents
\listoffigures
\listoftables

