\chapter{Cox Proportional Hazards Model}
\label{cox_proportional_hazards_model}

\section{Introduction}
\begin{itemize}
    \item Cox proportional hazards model - what is it?
    \item What is my hypothesis: using clinical and imaging features together in a simple cox regression will need to hugher concordance than imaging and clinical data alone
    \item What is concordance? How is it calculated?
    \item Examples of cox models in ALS before:
\end{itemize}

\section{Methods}
\begin{itemize}
    \item Clinical Data:
    \begin{itemize}
        \item Milan data: sample size, some partients died and some were censored and we're given the censor date
        \item Features: age at visit, alsfrsr, diagnostic delay, age at diagnosis. Imaging
        \item What preprocessing was done:
    \end{itemize}
    \item Imaging Data:
    \begin{itemize}
        \item MRI within 12 months of diagnosis
        \item Synthsegged into regions - give the regions
        \item Normalised (z-score I think, check this)
        \item Not all the regions were input into the model - give the regions and why (too many features)
    \end{itemize}
    \item Demographics of the sample: table
    \item Experiment details: how the cox model was trained and tested - python package, parameter choices, etc
    \begin{itemize}
        \item Kaplan meier plots: tertile split
        \item Stratified by sex
    \end{itemize}
\end{itemize}

\section{Results}

Clinical-only:
\begin{itemize}
    \item Table of results with all results in it
    \item What are the significant features and their p-values: none
    \item Hazard ratio plot
    \item Kaplan Meier plot
\end{itemize}

Imaging-only:
\begin{itemize}
    \item Table of results with all results in it
    \item What are the significant features and their p-values: lateral ventricle and brain stem
    \item Hazard ratio plot
    \item Kaplan Meier plot
\end{itemize}

Clinical and imaging:
\begin{itemize}
    \item Table of results with all results in it
    \item What are the significant features and their p-values: ALSFRSr, brain stem, diagnostic delay
    \item Hazard ratio plot
    \item Kaplan Meier plot
\end{itemize}

\section{Discussion}

Do these results make sense?:
\begin{itemize}
    \item Clinical-only: interesting no significant features, maybe not enough features
    \item Imaging-only: both significant features and their hazard ratios make sense given the literature (citations)
    \item Clinical and imaging: significant features and their hazard ratios make sense given the literature (citations)
\end{itemize}

Differences between inputs:
\begin{itemize}
    \item Higher concordance with clinical and imaging together as opposed to each by themselves. Better AIC as well.
    \item Different significant features for clinical and imaging alone and together. Why?
\end{itemize}

Limitations:
\begin{itemize}
    \item Small-ish sample size
    \item However, colinearity of features doesn't diminish the results
\end{itemize}

Take home message:
\begin{itemize}
    \item Using imaging-derived feature with clinical features increases concordance, more information into the model is better, makes the clinical measure significant as well when it wasn't before
    \item Would this result also be sustained when using more sophisticated machine learning models? Instinct says yes because they can handle more features and more complex relationships between features
\end{itemize}

\section{Conclusion}
\begin{itemize}
    \item Link to next chapter: we've shown that using imaging and clinical together is better than using them alone
    \item Let's see if we can improve on this by using more sophisticated machine learning models
\end{itemize}