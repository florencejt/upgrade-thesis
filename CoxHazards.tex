\chapter{Cox Proportional Hazards Model}
\label{cox_proportional_hazards_model}

\section{Introduction}
\begin{itemize}
    \item Cox proportional hazards model - what is it?
    \item What is my hypothesis: using clinical and imaging features together in a simple cox regression will need to hugher concordance than imaging and clinical data alone
    \item What is concordance? How is it calculated?
    \item Examples of cox models in ALS before:
\end{itemize}

\section{Data}

Data
- who, when, where
- dependent variable: time to death, censored by date of censorship if given or date of last data update

Clinical data - which features included, how were they preprocessed

Imaging data - which features included, how were they preprocessed
\begin{itemize}
    \item MRI within 12 months of diagnosis
    \item Synthsegged into regions - give the regions
    \item Normalised (z-score I think, check this)
    \item Not all the regions were input into the model - give the regions and why (too many features)
\end{itemize}

Demographics

\section{Methods}

What model am I using: Cox proportional hazards model

Python package lifelines

What experiments am I running? univariable, clinical only, imaging only, clinical and imaging

How am I assessing proportionality of hazards: Schoenfeld residuals

How am I assessing fit: concordance, AIC, log-likelihood

\section{Results}

Univariable
- list the significant features and their HRs and CIs

Clinical-only:
\begin{itemize}
    \item Including all clinical features: list the results and say that proportional hazards were broken by diagnosis age (p=0.002) and diagnostic delay (p=0.038)
    \item Doing univariable screening gives same significant factors and same broken assumptions
    \item Hazard ratio plot
    \item Kaplan Meier plot
\end{itemize}

Imaging-only:
\begin{itemize}
    \item With all features - proportional hazards broken by CSF (p=0.005) and cerebral white matter (p=0.0325)
    \item List significant features
    \item Significant features include cerebral white matter which was not significant in the unvariable model
    \item Because of high colinearity, we also ran the model with only the univariable significant features - no proportional hazards broken
    \item Link to table with those results and list new significant features: brain stem ,csf, amygdala, lateral ventricles, caudate, thalamus
    \item Hazard ratio plot
    \item Kaplan Meier plot
\end{itemize}

Clinical and imaging:
\begin{itemize}
    \item Significant features and their hazard ratios for all features included - alsfrsr, als, cerebral white matter (harmful)
    \item Proportional hazards broken by sex, bulbar onset, and cwm - all of which are univariable insignificant
    \item Ran again with uni significant features - no proportional hazards broken
    \item ALSFRS-R, ALS, amygdala, lateral ventricles
    \item Hazard ratio plot
    \item Kaplan Meier plot
\end{itemize}

\section{Discussion}

Univariable model:
\begin{itemize}
    \item Discuss the logic of the significant features
    \item What are the limitations of a univariable model - doesn't take into account the relationships between features
\end{itemize}

Clinical model:
\begin{itemize}
    \item Proportional hazards were broken by diagnosis age and diagnostic delay
    \item Why are we ok with this? Because we have a relatively small sample size, stratifying the model may lead to worse results than just including the features and acknowledging the violation
    \item And the hazard directions are consistent with the literature, so the model isn't completely wrong
    \item Talk about the significant features and their hazard ratios - do they make sense
    \item What about features that weren't significant - surprising? Not surprising? Look at demographics table
\end{itemize}

Imaging model:
\begin{itemize}
    \item Significant features include cerebral white matter which was not significant in the unvariable model
    \item With the univariable model, more features were significant, and all were clinically intuitive apart from thalamus
    \item Protective in univariable, hazard in multivariable - why? could be to do with the high collinearity still in the model
\end{itemize}

Clinical and imaging model:
\begin{itemize}
    \item proportional hazards fixed with uni significant features and resulting features are all clinically intuitive, although amygdala being significant over more intuitive features is surprising
    \item Two clinical and two imaging features were significant - shows that the imaging features are adding information to the model
    \item Talk about each significant feature
    \item Fit statistics: fit improved with multimodal features - this could be just because we're adding more features, but it could also be because we're adding more information
\end{itemize}

Limitations:
\begin{itemize}
    \item Small-ish sample size
    \item Multi-site without harmonisation
    \item Univariable screening - a controversial approach.
    \item However, colinearity of features doesn't diminish the results
\end{itemize}

Take home message:
\begin{itemize}
    \item Using imaging-derived feature with clinical features increases concordance, more information into the model is better, makes the clinical measure significant as well when it wasn't before
    \item Would this result also be sustained when using more sophisticated machine learning models? Instinct says yes because they can handle more features and more complex relationships between features
\end{itemize}

\section{Conclusion}
\begin{itemize}
    \item Link to next chapter: we've shown that using imaging and clinical together is better than using them alone
    \item Let's see if we can improve on this by using more sophisticated machine learning models
\end{itemize}