\chapter{Cox Proportional Hazards Model}
\label{cox_proportional_hazards_model}

\section{Introduction}

Many of the clinical and imaging features that were found to be associated with survival in Chapter~\ref{literature_review} were found using the Cox proportional hazards (CPH) model.
CPH is a survival analysis model that is used to investigate the relationship between the time to an event and the factors that may influence it.

As a first step in my investigation into the predictive power of clinical and imaging features in MND, I used a CPH model to investigate the relationship between the time to death and the clinical and neuroimaging features that were extracted from the ALS Biomarkers Study and Ospedale San Raffaele MND cohorts.

Previously, Querin and colleagues used a multivariable CPH to compare the predictive power of clinical and spinal cord imaging features in ALS in a cohort of 49 ALS patients, and concluded that spinal MRI measures were more predictive than clinical measures~\cite{querinSpinalCordMultiparametric2017}.
In this chapter, univariable, unimodal multivariable, and multimodal multivariable CPH models were used to investigate the relationship between the time to death and the clinical and imaging features in a larger cohort of 125 MND patients.

%\begin{itemize}
%    \item Cox proportional hazards model - what is it?
%    \item What is my hypothesis: using clinical and imaging features together in a simple cox regression will need to hugher concordance than imaging and clinical data alone
%    \item What is concordance? How is it calculated?
%    \item Examples of cox models in ALS before:
%\end{itemize}

\section{Data}

Data from two studies was used for this survival analysis: ALS Biomarkers Study and Opsedale San Raffaele.
Both of these datasets contain clinical information on MND patients and structural imaging conducted during their disease course.

The outcome of interest in this analysis is time to death, censored by date of censorship if recorded in the dataset or date of last data update if not.
The Ospedale San Raffaele cohort defines their endpoint as death or tracheostomy, but the ALS Biomarkers Study only defines their endpoint as death.

The clinical features in this study are the patient's sex, baseline ALSFRS-R, diagnostic delay, age at diagnosis, site of onset (categorised as bulbar and non-bulbar), signs of FTD, and their MND subtype (categorised as ALS and non-ALS).
These features were chosen for their clinical relevance to survival in MND and their availability in both datasets.

Patients were excluded if they did not have a T1- or T2-weighted MRI within 12 months before or after an MND diagnosis.
Regional brain volumes were extracted from the MRI using SynthSeg~\cite{billotSynthSegDomainRandomisation2021}, a modality-agnostic deep-learning segmentation tool.
A modality agnostic tool was chosen to overcome the inconsistency in MRI protocols within the ALS Biomarkers Study and between the ALS Biomarkers Study and Ospedale San Raffaele's MND cohort.
The dimensionality of the 33 regions was reduced to 12 by summing right and left regions and choosing regions relevant to MND pathology.
The remaining regions were z-score normalised.

\begin{table}
\centering
\caption{Demographics and clinical characteristics of the patients included in the analysis. ALSFRSr: ALS Functional Rating Scale-Revised.}
\label{tab:coxdemographics}
\begin{tabular}{|ll|}
\hline
              \textbf{Variable}                          &      \\
\hline
 n                               & 125         \\\hline
 Sex, n (\%)          & \\
 \hspace{5mm}Female & 55 (44.0)   \\
\hspace{5mm}Male & 70 (56.0)   \\\hline
 ALSFRSr, mean (SD)                 & 38.9 (6.4)  \\\hline
 Survival (months), mean (SD)         & 34.3 (27.3) \\\hline
 Diagnostic Delay (months), mean (SD)  & 13.6 (13.2) \\\hline
 Age at Diagnosis (years), mean (SD)  & 62.6 (11.4) \\\hline
 Site of Onset, n (\%)        &  \\
 \hspace{5mm}Non-bulbar   & 91 (72.8)   \\
 \hspace{5mm}Bulbar  & 34 (27.2)   \\\hline
 Frontotemporal Dementia, n (\%)       & \\
 \hspace{5mm}Not present & 87 (69.6)   \\
\hspace{5mm}Present & 38 (30.4)   \\\hline
 MND Type, n (\%)                & \\
 \hspace{5mm}Not ALS   & 21 (16.8)   \\
 \hspace{5mm}ALS   & 104 (83.2)  \\\hline
 Outcome, n (\%)                          & \\
 \hspace{5mm}Censored   & 21 (16.8)   \\
\hspace{5mm}Died   & 104 (83.2)  \\\hline

\end{tabular}
\end{table}

The demographics and clinical characteristics of the patients who were included in the analysis are shown in Table~\ref{tab:coxdemographics}.
Out of the 125 patients, 21 were censored and 104 died during the study period.
Censored patients were those who were still alive at the end of the study period (or at the time of data download: 5th March 2024) or who were lost to follow-up before the end of the study period.

\section{Methods}

A CPH model is expressed as
\begin{equation}\label{eq:coxhazard}
    h(t) = h_0(t) \exp{(\beta_1 X_1 + \beta_2 X_2 + ... + \beta_n X_n)},
\end{equation}
where $t$ is the survival time, $h(t)$ is the hazard function, $\bold{X}$ are the variables investigated, $h_0(t)$ is the baseline hazard if all the variables were 0.
The hazard ratios (HRs) are represented as $\beta$, with $\beta_1$ being the hazard ratio for the variable $X_1$.
A HR larger than 1 indicates the variable is associated with a poorer prognosis, or increased risk of the event, which is death in this CPH application.
The assumption of proportional hazards is that each individual in the analysis has the same hazard function, but the hazard functions are scaled by a constant factor, which is not dependent on time.
The Python package, lifelines, was used to implement the CPH models and to test the proportional hazards assumption (citation here).
Since the aim of this analysis was to see how multimodal data affects survival analysis of MND, four models were run:
\begin{enumerate}
\setlength\itemsep{-0.5em}
    \item Univariable model: each feature was tested individually to see if it was associated with survival without adjusting for other features.
    \item Clinical-only model: multivariable model with only clinical features.
    \item Imaging-only model: multivariable model with only imaging features.
    \item Clinical and imaging model: multivariable model with both clinical and imaging features.
\end{enumerate}

%The models we used were not stratified by any variables, as the sample size was relatively small and stratifying the model may lead to less informative results than just including the features and acknowledging the violation of the proportional hazards assumption.

The CPH models estimated the hazard ratios and their 95\% confidence intervals for each variable, with significance level set at $p<0.05$.

The quality of models fit to the data was assessed using the concordance index, which is a measure of how well the model predicts the order of the survival times, and the Akaike Information Criterion (AIC), which is a measure of the model's goodness of fit balanced with its complexity.

\section{Results}

\subsection{Univariable}

% \begin{table}
% \centering
% % \resizebox{1.2\linewidth}{!}{%
% \begin{sidewaystable}
% \begin{tabular}{|l|ll|ll|ll|ll|} 
% \cline{2-9}
% \multicolumn{1}{l|}{} & \multicolumn{2}{l|}{\multirow{2}{*}{Univariable}} & \multicolumn{6}{l|}{Multivariable} \\ 
% \cline{4-9}
% \multicolumn{1}{l|}{} & \multicolumn{2}{l|}{} & \multicolumn{2}{l|}{Clinical} & \multicolumn{2}{l|}{Imaging} & \multicolumn{2}{l|}{Multimodal} \\ 
% \hline
% Variable & HR (95\% CI) & $p$ & HR (95\% CI) & $p$ & HR (95\% CI) & $p$ & HR (95\% CI) & $p$ \\ 
% \hline
% Sex &  &  &  &  & {\cellcolor[rgb]{0.753,0.753,0.753}} & {\cellcolor[rgb]{0.753,0.753,0.753}} &  &  \\
% Female & 1.00, Ref & - & 1.00, Ref & - & {\cellcolor[rgb]{0.753,0.753,0.753}} & {\cellcolor[rgb]{0.753,0.753,0.753}} & 1.00, Ref & - \\
% Male & 0.91 (0.62--1.34) & 0.6417 & \textcolor[rgb]{0.2,0.2,0.2}{1.13 (0.72 -- 1.77)} & \textcolor[rgb]{0.2,0.2,0.2}{0.5995} & {\cellcolor[rgb]{0.753,0.753,0.753}} & {\cellcolor[rgb]{0.753,0.753,0.753}} & \textcolor[rgb]{0.2,0.2,0.2}{1.11 (0.62 -- 1.99)} & \textcolor[rgb]{0.2,0.2,0.2}{0.7315} \\ 
% \hline
% ALSFRS-R & 0.70 (0.60--0.82) & \begin{tabular}[c]{@{}l@{}}$<$\textbf{\textbf{0.0001}}\\\end{tabular} & \textcolor[rgb]{0.2,0.2,0.2}{0.62 (0.51 -- 0.76)} & $<$\textbf{0.0001} & {\cellcolor[rgb]{0.753,0.753,0.753}} & {\cellcolor[rgb]{0.753,0.753,0.753}} & \textcolor[rgb]{0.2,0.2,0.2}{0.68 (0.53 -- 0.88)} & \textcolor[rgb]{0.2,0.2,0.2}{\textbf{0.0034}} \\ 
% \hline
% Diagnostic delay, mo & 0.77 (0.60--0.99) & \textbf{0.0409} & \textcolor[rgb]{0.2,0.2,0.2}{0.79 (0.59 -- 1.06)} & \textcolor[rgb]{0.2,0.2,0.2}{0.1165} & {\cellcolor[rgb]{0.753,0.753,0.753}} & {\cellcolor[rgb]{0.753,0.753,0.753}} & \textcolor[rgb]{0.2,0.2,0.2}{0.86 (0.63 -- 1.18)} & \textcolor[rgb]{0.2,0.2,0.2}{0.3500} \\ 
% \hline
% Age at diagnosis, yr & 1.48 (1.29--1.84) & \textbf{0.0005} & \textcolor[rgb]{0.2,0.2,0.2}{1.53 (1.23 -- 1.92)} & \textcolor[rgb]{0.2,0.2,0.2}{\textbf{0.0002}} & {\cellcolor[rgb]{0.753,0.753,0.753}} & {\cellcolor[rgb]{0.753,0.753,0.753}} & \textcolor[rgb]{0.2,0.2,0.2}{1.19 (0.84 -- 1.67)} & \textcolor[rgb]{0.2,0.2,0.2}{0.3265} \\ 
% \hline
% Site of onset &  &  &  &  & {\cellcolor[rgb]{0.753,0.753,0.753}} & {\cellcolor[rgb]{0.753,0.753,0.753}} &  &  \\
% Non-bulbar & 1.00, Ref & - & 1.00, Ref & - & {\cellcolor[rgb]{0.753,0.753,0.753}} & {\cellcolor[rgb]{0.753,0.753,0.753}} & 1.00, Ref & - \\
% Bulbar & 1.36 (0.88--2.10) & 0.1605 & \textcolor[rgb]{0.2,0.2,0.2}{0.90 (0.55 -- 1.47)} & \textcolor[rgb]{0.2,0.2,0.2}{0.6695} & {\cellcolor[rgb]{0.753,0.753,0.753}} & {\cellcolor[rgb]{0.753,0.753,0.753}} & \textcolor[rgb]{0.2,0.2,0.2}{0.94 (0.54 -- 1.64)} & \textcolor[rgb]{0.2,0.2,0.2}{0.8280} \\ 
% \hline
% FTD &  &  &  &  & {\cellcolor[rgb]{0.753,0.753,0.753}} & {\cellcolor[rgb]{0.753,0.753,0.753}} &  &  \\
% No & 1.00, Ref & - & 1.00, Ref & - & {\cellcolor[rgb]{0.753,0.753,0.753}} & {\cellcolor[rgb]{0.753,0.753,0.753}} & 1.00, Ref & - \\
% Yes & 1.58 (1.04--2.41) & \textbf{0.0337} & \textcolor[rgb]{0.2,0.2,0.2}{1.46 (0.91 -- 2.35)} & \textcolor[rgb]{0.2,0.2,0.2}{0.1152} & {\cellcolor[rgb]{0.753,0.753,0.753}} & {\cellcolor[rgb]{0.753,0.753,0.753}} & \textcolor[rgb]{0.2,0.2,0.2}{1.20 (0.69 -- 2.08)} & \textcolor[rgb]{0.2,0.2,0.2}{0.5177} \\ 
% \hline
% MND Subtype &  &  &  &  & {\cellcolor[rgb]{0.753,0.753,0.753}} & {\cellcolor[rgb]{0.753,0.753,0.753}} &  &  \\
% Non-ALS & 1.00, Ref & - & 1.00, Ref & - & {\cellcolor[rgb]{0.753,0.753,0.753}} & {\cellcolor[rgb]{0.753,0.753,0.753}} & 1.00, Ref & - \\
% ALS & 2.40 (1.36--4.23) & \textbf{0.0026} & \textcolor[rgb]{0.2,0.2,0.2}{1.96 (1.04 -- 3.71)} & \textcolor[rgb]{0.2,0.2,0.2}{\textbf{0.0384}} & {\cellcolor[rgb]{0.753,0.753,0.753}} & {\cellcolor[rgb]{0.753,0.753,0.753}} & \textcolor[rgb]{0.2,0.2,0.2}{2.32 (1.10 -- 4.88)} & \textcolor[rgb]{0.2,0.2,0.2}{\textbf{0.0264}} \\ 
% \hline
% Brain stem & \textcolor[rgb]{0.2,0.2,0.2}{0.66 (0.53 -- 0.83)} & \textcolor[rgb]{0.2,0.2,0.2}{\textbf{0.0003}} & {\cellcolor[rgb]{0.753,0.753,0.753}} & {\cellcolor[rgb]{0.753,0.753,0.753}} & \textcolor[rgb]{0.2,0.2,0.2}{0.64 (0.41 -- 1.01)} & \textcolor[rgb]{0.2,0.2,0.2}{0.0557} & \textcolor[rgb]{0.2,0.2,0.2}{0.64 (0.38 -- 1.08)} & \textcolor[rgb]{0.2,0.2,0.2}{0.0958} \\ 
% \hline
% CSF & \textcolor[rgb]{0.2,0.2,0.2}{1.40 (1.16 -- 1.70)} & \textcolor[rgb]{0.2,0.2,0.2}{\textbf{0.0005}} & {\cellcolor[rgb]{0.753,0.753,0.753}} & {\cellcolor[rgb]{0.753,0.753,0.753}} & \textcolor[rgb]{0.2,0.2,0.2}{1.10 (0.79 -- 1.54)} & \textcolor[rgb]{0.2,0.2,0.2}{0.5794} & \textcolor[rgb]{0.2,0.2,0.2}{0.93 (0.61 -- 1.43)} & \textcolor[rgb]{0.2,0.2,0.2}{0.7370} \\ 
% \hline
% Lateral ventricles & \textcolor[rgb]{0.2,0.2,0.2}{1.58 (1.32 -- 1.89)} & \textbf{$<$0.0001} & {\cellcolor[rgb]{0.753,0.753,0.753}} & {\cellcolor[rgb]{0.753,0.753,0.753}} & \textcolor[rgb]{0.2,0.2,0.2}{1.52 (1.06 -- 2.17)} & \textcolor[rgb]{0.2,0.2,0.2}{\textbf{0.0214}} & \textcolor[rgb]{0.2,0.2,0.2}{1.47 (0.96 -- 2.24)} & \textcolor[rgb]{0.2,0.2,0.2}{0.0731} \\ 
% \hline
% Hippocampus & \textcolor[rgb]{0.2,0.2,0.2}{0.59 (0.47 -- 0.73)} & \textbf{$<$0.0001} & {\cellcolor[rgb]{0.753,0.753,0.753}} & {\cellcolor[rgb]{0.753,0.753,0.753}} & \textcolor[rgb]{0.2,0.2,0.2}{0.96 (0.58 -- 1.61)} & \textcolor[rgb]{0.2,0.2,0.2}{0.8811} & \textcolor[rgb]{0.2,0.2,0.2}{1.18 (0.67 -- 2.08)} & \textcolor[rgb]{0.2,0.2,0.2}{0.5663} \\ 
% \hline
% Amygdala & \textcolor[rgb]{0.2,0.2,0.2}{0.59 (0.47 -- 0.73)} & \textbf{$<$0.0001} & {\cellcolor[rgb]{0.753,0.753,0.753}} & {\cellcolor[rgb]{0.753,0.753,0.753}} & \textcolor[rgb]{0.2,0.2,0.2}{0.65 (0.42 -- 1.00)} & \textcolor[rgb]{0.2,0.2,0.2}{0.0502} & \textcolor[rgb]{0.2,0.2,0.2}{0.68 (0.43 -- 1.07)} & \textcolor[rgb]{0.2,0.2,0.2}{0.0927} \\ 
% \hline
% Thalamus & \textcolor[rgb]{0.2,0.2,0.2}{0.78 (0.64 -- 0.96)} & \textcolor[rgb]{0.2,0.2,0.2}{\textbf{0.0173}} & {\cellcolor[rgb]{0.753,0.753,0.753}} & {\cellcolor[rgb]{0.753,0.753,0.753}} & \textcolor[rgb]{0.2,0.2,0.2}{1.30 (0.79 -- 2.14)} & \textcolor[rgb]{0.2,0.2,0.2}{0.2931} & \textcolor[rgb]{0.2,0.2,0.2}{1.14 (0.67 -- 1.93)} & \textcolor[rgb]{0.2,0.2,0.2}{0.6229} \\ 
% \hline
% Caudate & \textcolor[rgb]{0.2,0.2,0.2}{0.77 (0.64 -- 0.93)} & \textcolor[rgb]{0.2,0.2,0.2}{\textbf{0.0052}} & {\cellcolor[rgb]{0.753,0.753,0.753}} & {\cellcolor[rgb]{0.753,0.753,0.753}} & \textcolor[rgb]{0.2,0.2,0.2}{0.62 (0.41 -- 0.94)} & \textcolor[rgb]{0.2,0.2,0.2}{\textbf{0.0245}} & \textcolor[rgb]{0.2,0.2,0.2}{0.69 (0.43 -- 1.11)} & \textcolor[rgb]{0.2,0.2,0.2}{0.1253} \\ 
% \hline
% Putamen & \textcolor[rgb]{0.2,0.2,0.2}{0.72 (0.60 -- 0.87)} & \textcolor[rgb]{0.2,0.2,0.2}{\textbf{0.0007}} & {\cellcolor[rgb]{0.753,0.753,0.753}} & {\cellcolor[rgb]{0.753,0.753,0.753}} & \textcolor[rgb]{0.2,0.2,0.2}{1.30 (0.71 -- 2.37)} & \textcolor[rgb]{0.2,0.2,0.2}{0.3884} & \textcolor[rgb]{0.2,0.2,0.2}{0.88 (0.47 -- 1.66)} & \textcolor[rgb]{0.2,0.2,0.2}{0.6932} \\ 
% \hline
% Pallidum & \textcolor[rgb]{0.2,0.2,0.2}{0.75 (0.61 -- 0.91)} & \textcolor[rgb]{0.2,0.2,0.2}{\textbf{0.0041}} & {\cellcolor[rgb]{0.753,0.753,0.753}} & {\cellcolor[rgb]{0.753,0.753,0.753}} & \textcolor[rgb]{0.2,0.2,0.2}{0.75 (0.50 -- 1.14)} & \textcolor[rgb]{0.2,0.2,0.2}{0.1819} & \textcolor[rgb]{0.2,0.2,0.2}{0.87 (0.57 -- 1.32)} & \textcolor[rgb]{0.2,0.2,0.2}{0.5007} \\ 
% \hline
% Cerebral white matter & \textcolor[rgb]{0.2,0.2,0.2}{0.99 (0.82 -- 1.20)} & \textcolor[rgb]{0.2,0.2,0.2}{0.9339} & {\cellcolor[rgb]{0.753,0.753,0.753}} & {\cellcolor[rgb]{0.753,0.753,0.753}} & \textcolor[rgb]{0.2,0.2,0.2}{2.35 (1.16 -- 4.77)} & \textcolor[rgb]{0.2,0.2,0.2}{\textbf{0.0181}} & \textcolor[rgb]{0.2,0.2,0.2}{2.43 (1.16 -- 5.10)} & \textcolor[rgb]{0.2,0.2,0.2}{\textbf{0.0185}} \\ 
% \hline
% Cerebellum white matter & \textcolor[rgb]{0.2,0.2,0.2}{0.83 (0.67 -- 1.02)} & \textcolor[rgb]{0.2,0.2,0.2}{0.0775} & {\cellcolor[rgb]{0.753,0.753,0.753}} & {\cellcolor[rgb]{0.753,0.753,0.753}} & \textcolor[rgb]{0.2,0.2,0.2}{0.97 (0.56 -- 1.66)} & \textcolor[rgb]{0.2,0.2,0.2}{0.9054} & \textcolor[rgb]{0.2,0.2,0.2}{1.14 (0.63 -- 2.05)} & \textcolor[rgb]{0.2,0.2,0.2}{0.6714} \\ 
% \hline
% Cerebellum cortex & \textcolor[rgb]{0.2,0.2,0.2}{0.71 (0.57 -- 0.88)} & \textcolor[rgb]{0.2,0.2,0.2}{\textbf{0.0016}} & {\cellcolor[rgb]{0.753,0.753,0.753}} & {\cellcolor[rgb]{0.753,0.753,0.753}} & \textcolor[rgb]{0.2,0.2,0.2}{0.79 (0.52 -- 1.21)} & \textcolor[rgb]{0.2,0.2,0.2}{0.2888} & \textcolor[rgb]{0.2,0.2,0.2}{0.82 (0.51 -- 1.32)} & \textcolor[rgb]{0.2,0.2,0.2}{0.4135} \\ 
% \hline
% Cerebral cortex & \textcolor[rgb]{0.2,0.2,0.2}{0.99 (0.82 -- 1.21)} & \textcolor[rgb]{0.2,0.2,0.2}{0.9462} & {\cellcolor[rgb]{0.753,0.753,0.753}} & {\cellcolor[rgb]{0.753,0.753,0.753}} & \textcolor[rgb]{0.2,0.2,0.2}{0.90 (0.45 -- 1.83)} & \textcolor[rgb]{0.2,0.2,0.2}{0.7769} & \textcolor[rgb]{0.2,0.2,0.2}{0.87 (0.41 -- 1.84)} & \textcolor[rgb]{0.2,0.2,0.2}{0.7118} \\
% \hline
% \end{tabular}
% \end{sidewaystable}

% \end{table}

% Please add the following required packages to your document preamble:
% \usepackage{multirow}
% \usepackage{graphicx}
% \usepackage[table,xcdraw]{xcolor}
% Beamer presentation requires \usepackage{colortbl} instead of \usepackage[table,xcdraw]{xcolor}
% \usepackage{multirow}
% \usepackage{colortbl}
% \usepackage{rotating}

\begin{sidewaystable}
{\small\setlength{\tabcolsep}{4pt}
\centering
\caption{Hazard ratios of survival risk in patients with motor neuron disease for univariable and three multivariable Cox proportional hazards regressions: clinical only, imaging-features only, and clinical and imaging features together (multimodal). Acronyms: FTD - frontotemporal dementia, ALSFRS-R - revised amytrophic lateral sclerosis functional rating scale, CSF - cerebrospinal fluid.}
\label{tab:allfeatures_cox}
\begin{tabular}{|l|lr|lr|lr|lr|} 
\cline{2-9}
\multicolumn{1}{l|}{} & \multicolumn{2}{l|}{\multirow{2}{*}{Univariable}} & \multicolumn{6}{c|}{Multivariable} \\ 
\cline{4-9}
\multicolumn{1}{l|}{} & \multicolumn{2}{l|}{} & \multicolumn{2}{l|}{Clinical} & \multicolumn{2}{l|}{Imaging} & \multicolumn{2}{l|}{Multimodal} \\ 
\hline
Variable & HR (95\% CI) & $p$ & HR (95\% CI) & $p$ & HR (95\% CI) & $p$ & HR (95\% CI) & $p$ \\ \hline
\multicolumn{1}{|l}{\textbf{Clinical}} &  & \multicolumn{1}{l}{} &  & \multicolumn{1}{l}{} &  & \multicolumn{1}{l}{} &  &  \\ 
\hline
Sex &  &  &  &  & {\cellcolor[rgb]{0.753,0.753,0.753}} & {\cellcolor[rgb]{0.753,0.753,0.753}} &  &  \\
\hspace{5mm}Female & 1.00, Ref & - & 1.00, Ref & - & {\cellcolor[rgb]{0.753,0.753,0.753}} & {\cellcolor[rgb]{0.753,0.753,0.753}} & 1.00, Ref & - \\
\hspace{5mm}Male & 0.91 (0.62--1.34) & 0.6417 & \textcolor[rgb]{0.2,0.2,0.2}{1.13 (0.72 -- 1.77)} & \textcolor[rgb]{0.2,0.2,0.2}{0.5995} & {\cellcolor[rgb]{0.753,0.753,0.753}} & {\cellcolor[rgb]{0.753,0.753,0.753}} & \textcolor[rgb]{0.2,0.2,0.2}{1.11 (0.62 -- 1.99)} & \textcolor[rgb]{0.2,0.2,0.2}{0.7315} \\ 
\hline
ALSFRS-R & 0.70 (0.60--0.82) & \begin{tabular}[c]{@{}l@{}}$<$\textbf{\textbf{0.0001}}\\\end{tabular} & \textcolor[rgb]{0.2,0.2,0.2}{0.62 (0.51 -- 0.76)} & $<$\textbf{0.0001} & {\cellcolor[rgb]{0.753,0.753,0.753}} & {\cellcolor[rgb]{0.753,0.753,0.753}} & \textcolor[rgb]{0.2,0.2,0.2}{0.68 (0.53 -- 0.88)} & \textcolor[rgb]{0.2,0.2,0.2}{\textbf{0.0034}} \\ 
\hline
Diagnostic delay, mo & 0.77 (0.60--0.99) & \textbf{0.0409} & \textcolor[rgb]{0.2,0.2,0.2}{0.79 (0.59 -- 1.06)} & \textcolor[rgb]{0.2,0.2,0.2}{0.1165} & {\cellcolor[rgb]{0.753,0.753,0.753}} & {\cellcolor[rgb]{0.753,0.753,0.753}} & \textcolor[rgb]{0.2,0.2,0.2}{0.86 (0.63 -- 1.18)} & \textcolor[rgb]{0.2,0.2,0.2}{0.3500} \\ 
\hline
Age at diagnosis, yr & 1.48 (1.29--1.84) & \textbf{0.0005} & \textcolor[rgb]{0.2,0.2,0.2}{1.53 (1.23 -- 1.92)} & \textcolor[rgb]{0.2,0.2,0.2}{\textbf{0.0002}} & {\cellcolor[rgb]{0.753,0.753,0.753}} & {\cellcolor[rgb]{0.753,0.753,0.753}} & \textcolor[rgb]{0.2,0.2,0.2}{1.19 (0.84 -- 1.67)} & \textcolor[rgb]{0.2,0.2,0.2}{0.3265} \\ 
\hline
Site of onset &  &  &  &  & {\cellcolor[rgb]{0.753,0.753,0.753}} & {\cellcolor[rgb]{0.753,0.753,0.753}} &  &  \\
\hspace{5mm}Non-bulbar & 1.00, Ref & - & 1.00, Ref & - & {\cellcolor[rgb]{0.753,0.753,0.753}} & {\cellcolor[rgb]{0.753,0.753,0.753}} & 1.00, Ref & - \\
\hspace{5mm}Bulbar & 1.36 (0.88--2.10) & 0.1605 & \textcolor[rgb]{0.2,0.2,0.2}{0.90 (0.55 -- 1.47)} & \textcolor[rgb]{0.2,0.2,0.2}{0.6695} & {\cellcolor[rgb]{0.753,0.753,0.753}} & {\cellcolor[rgb]{0.753,0.753,0.753}} & \textcolor[rgb]{0.2,0.2,0.2}{0.94 (0.54 -- 1.64)} & \textcolor[rgb]{0.2,0.2,0.2}{0.8280} \\ 
\hline
FTD &  &  &  &  & {\cellcolor[rgb]{0.753,0.753,0.753}} & {\cellcolor[rgb]{0.753,0.753,0.753}} &  &  \\
\hspace{5mm}No & 1.00, Ref & - & 1.00, Ref & - & {\cellcolor[rgb]{0.753,0.753,0.753}} & {\cellcolor[rgb]{0.753,0.753,0.753}} & 1.00, Ref & - \\
\hspace{5mm}Yes & 1.58 (1.04--2.41) & \textbf{0.0337} & \textcolor[rgb]{0.2,0.2,0.2}{1.46 (0.91 -- 2.35)} & \textcolor[rgb]{0.2,0.2,0.2}{0.1152} & {\cellcolor[rgb]{0.753,0.753,0.753}} & {\cellcolor[rgb]{0.753,0.753,0.753}} & \textcolor[rgb]{0.2,0.2,0.2}{1.20 (0.69 -- 2.08)} & \textcolor[rgb]{0.2,0.2,0.2}{0.5177} \\ 
\hline
MND Subtype &  &  &  &  & {\cellcolor[rgb]{0.753,0.753,0.753}} & {\cellcolor[rgb]{0.753,0.753,0.753}} &  &  \\
\hspace{5mm}Non-ALS & 1.00, Ref & - & 1.00, Ref & - & {\cellcolor[rgb]{0.753,0.753,0.753}} & {\cellcolor[rgb]{0.753,0.753,0.753}} & 1.00, Ref & - \\
\hspace{5mm}ALS & 2.40 (1.36--4.23) & \textbf{0.0026} & \textcolor[rgb]{0.2,0.2,0.2}{1.96 (1.04 -- 3.71)} & \textcolor[rgb]{0.2,0.2,0.2}{\textbf{0.0384}} & {\cellcolor[rgb]{0.753,0.753,0.753}} & {\cellcolor[rgb]{0.753,0.753,0.753}} & \textcolor[rgb]{0.2,0.2,0.2}{2.32 (1.10 -- 4.88)} & \textcolor[rgb]{0.2,0.2,0.2}{\textbf{0.0264}} \\ 
\hline
\multicolumn{1}{|l}{\textbf{Volumes}} &  & \multicolumn{1}{l}{} &  & \multicolumn{1}{l}{} &  & \multicolumn{1}{l}{} &  &  \\ 
\hline
Brain stem & \textcolor[rgb]{0.2,0.2,0.2}{0.66 (0.53 -- 0.83)} & \textcolor[rgb]{0.2,0.2,0.2}{\textbf{0.0003}} & {\cellcolor[rgb]{0.753,0.753,0.753}} & {\cellcolor[rgb]{0.753,0.753,0.753}} & \textcolor[rgb]{0.2,0.2,0.2}{0.64 (0.41 -- 1.01)} & \textcolor[rgb]{0.2,0.2,0.2}{0.0557} & \textcolor[rgb]{0.2,0.2,0.2}{0.64 (0.38 -- 1.08)} & \textcolor[rgb]{0.2,0.2,0.2}{0.0958} \\ 
\hline
CSF & \textcolor[rgb]{0.2,0.2,0.2}{1.40 (1.16 -- 1.70)} & \textcolor[rgb]{0.2,0.2,0.2}{\textbf{0.0005}} & {\cellcolor[rgb]{0.753,0.753,0.753}} & {\cellcolor[rgb]{0.753,0.753,0.753}} & \textcolor[rgb]{0.2,0.2,0.2}{1.10 (0.79 -- 1.54)} & \textcolor[rgb]{0.2,0.2,0.2}{0.5794} & \textcolor[rgb]{0.2,0.2,0.2}{0.93 (0.61 -- 1.43)} & \textcolor[rgb]{0.2,0.2,0.2}{0.7370} \\ 
\hline
Lateral ventricles & \textcolor[rgb]{0.2,0.2,0.2}{1.58 (1.32 -- 1.89)} & \textbf{$<$0.0001} & {\cellcolor[rgb]{0.753,0.753,0.753}} & {\cellcolor[rgb]{0.753,0.753,0.753}} & \textcolor[rgb]{0.2,0.2,0.2}{1.52 (1.06 -- 2.17)} & \textcolor[rgb]{0.2,0.2,0.2}{\textbf{0.0214}} & \textcolor[rgb]{0.2,0.2,0.2}{1.47 (0.96 -- 2.24)} & \textcolor[rgb]{0.2,0.2,0.2}{0.0731} \\ 
\hline
Hippocampus & \textcolor[rgb]{0.2,0.2,0.2}{0.59 (0.47 -- 0.73)} & \textbf{$<$0.0001} & {\cellcolor[rgb]{0.753,0.753,0.753}} & {\cellcolor[rgb]{0.753,0.753,0.753}} & \textcolor[rgb]{0.2,0.2,0.2}{0.96 (0.58 -- 1.61)} & \textcolor[rgb]{0.2,0.2,0.2}{0.8811} & \textcolor[rgb]{0.2,0.2,0.2}{1.18 (0.67 -- 2.08)} & \textcolor[rgb]{0.2,0.2,0.2}{0.5663} \\ 
\hline
Amygdala & \textcolor[rgb]{0.2,0.2,0.2}{0.59 (0.47 -- 0.73)} & \textbf{$<$0.0001} & {\cellcolor[rgb]{0.753,0.753,0.753}} & {\cellcolor[rgb]{0.753,0.753,0.753}} & \textcolor[rgb]{0.2,0.2,0.2}{0.65 (0.42 -- 1.00)} & \textcolor[rgb]{0.2,0.2,0.2}{0.0502} & \textcolor[rgb]{0.2,0.2,0.2}{0.68 (0.43 -- 1.07)} & \textcolor[rgb]{0.2,0.2,0.2}{0.0927} \\ 
\hline
Thalamus & \textcolor[rgb]{0.2,0.2,0.2}{0.78 (0.64 -- 0.96)} & \textcolor[rgb]{0.2,0.2,0.2}{\textbf{0.0173}} & {\cellcolor[rgb]{0.753,0.753,0.753}} & {\cellcolor[rgb]{0.753,0.753,0.753}} & \textcolor[rgb]{0.2,0.2,0.2}{1.30 (0.79 -- 2.14)} & \textcolor[rgb]{0.2,0.2,0.2}{0.2931} & \textcolor[rgb]{0.2,0.2,0.2}{1.14 (0.67 -- 1.93)} & \textcolor[rgb]{0.2,0.2,0.2}{0.6229} \\ 
\hline
Caudate & \textcolor[rgb]{0.2,0.2,0.2}{0.77 (0.64 -- 0.93)} & \textcolor[rgb]{0.2,0.2,0.2}{\textbf{0.0052}} & {\cellcolor[rgb]{0.753,0.753,0.753}} & {\cellcolor[rgb]{0.753,0.753,0.753}} & \textcolor[rgb]{0.2,0.2,0.2}{0.62 (0.41 -- 0.94)} & \textcolor[rgb]{0.2,0.2,0.2}{\textbf{0.0245}} & \textcolor[rgb]{0.2,0.2,0.2}{0.69 (0.43 -- 1.11)} & \textcolor[rgb]{0.2,0.2,0.2}{0.1253} \\ 
\hline
Putamen & \textcolor[rgb]{0.2,0.2,0.2}{0.72 (0.60 -- 0.87)} & \textcolor[rgb]{0.2,0.2,0.2}{\textbf{0.0007}} & {\cellcolor[rgb]{0.753,0.753,0.753}} & {\cellcolor[rgb]{0.753,0.753,0.753}} & \textcolor[rgb]{0.2,0.2,0.2}{1.30 (0.71 -- 2.37)} & \textcolor[rgb]{0.2,0.2,0.2}{0.3884} & \textcolor[rgb]{0.2,0.2,0.2}{0.88 (0.47 -- 1.66)} & \textcolor[rgb]{0.2,0.2,0.2}{0.6932} \\ 
\hline
Pallidum & \textcolor[rgb]{0.2,0.2,0.2}{0.75 (0.61 -- 0.91)} & \textcolor[rgb]{0.2,0.2,0.2}{\textbf{0.0041}} & {\cellcolor[rgb]{0.753,0.753,0.753}} & {\cellcolor[rgb]{0.753,0.753,0.753}} & \textcolor[rgb]{0.2,0.2,0.2}{0.75 (0.50 -- 1.14)} & \textcolor[rgb]{0.2,0.2,0.2}{0.1819} & \textcolor[rgb]{0.2,0.2,0.2}{0.87 (0.57 -- 1.32)} & \textcolor[rgb]{0.2,0.2,0.2}{0.5007} \\ 
\hline
Cerebral white matter & \textcolor[rgb]{0.2,0.2,0.2}{0.99 (0.82 -- 1.20)} & \textcolor[rgb]{0.2,0.2,0.2}{0.9339} & {\cellcolor[rgb]{0.753,0.753,0.753}} & {\cellcolor[rgb]{0.753,0.753,0.753}} & \textcolor[rgb]{0.2,0.2,0.2}{2.35 (1.16 -- 4.77)} & \textcolor[rgb]{0.2,0.2,0.2}{\textbf{0.0181}} & \textcolor[rgb]{0.2,0.2,0.2}{2.43 (1.16 -- 5.10)} & \textcolor[rgb]{0.2,0.2,0.2}{\textbf{0.0185}} \\ 
\hline
Cerebellum white matter & \textcolor[rgb]{0.2,0.2,0.2}{0.83 (0.67 -- 1.02)} & \textcolor[rgb]{0.2,0.2,0.2}{0.0775} & {\cellcolor[rgb]{0.753,0.753,0.753}} & {\cellcolor[rgb]{0.753,0.753,0.753}} & \textcolor[rgb]{0.2,0.2,0.2}{0.97 (0.56 -- 1.66)} & \textcolor[rgb]{0.2,0.2,0.2}{0.9054} & \textcolor[rgb]{0.2,0.2,0.2}{1.14 (0.63 -- 2.05)} & \textcolor[rgb]{0.2,0.2,0.2}{0.6714} \\ 
\hline
Cerebellum cortex & \textcolor[rgb]{0.2,0.2,0.2}{0.71 (0.57 -- 0.88)} & \textcolor[rgb]{0.2,0.2,0.2}{\textbf{0.0016}} & {\cellcolor[rgb]{0.753,0.753,0.753}} & {\cellcolor[rgb]{0.753,0.753,0.753}} & \textcolor[rgb]{0.2,0.2,0.2}{0.79 (0.52 -- 1.21)} & \textcolor[rgb]{0.2,0.2,0.2}{0.2888} & \textcolor[rgb]{0.2,0.2,0.2}{0.82 (0.51 -- 1.32)} & \textcolor[rgb]{0.2,0.2,0.2}{0.4135} \\ 
\hline
Cerebral cortex & \textcolor[rgb]{0.2,0.2,0.2}{0.99 (0.82 -- 1.21)} & \textcolor[rgb]{0.2,0.2,0.2}{0.9462} & {\cellcolor[rgb]{0.753,0.753,0.753}} & {\cellcolor[rgb]{0.753,0.753,0.753}} & \textcolor[rgb]{0.2,0.2,0.2}{0.90 (0.45 -- 1.83)} & \textcolor[rgb]{0.2,0.2,0.2}{0.7769} & \textcolor[rgb]{0.2,0.2,0.2}{0.87 (0.41 -- 1.84)} & \textcolor[rgb]{0.2,0.2,0.2}{0.7118} \\
\hline
\end{tabular}
}
\end{sidewaystable}


% \usepackage{multirow}
% \usepackage{colortbl}
% \usepackage{rotating}


\begin{sidewaystable}
\centering
\caption{Univariable-screened hazard ratios of survival risk in patients with motor neuron disease for three multivariable Cox proportional hazards regressions: clinical only, imaging-features only, and clinical and imaging features together (multimodal). The features included are significant in univariable Cox regressions. Acronyms: FTD - frontotemporal dementia, ALSFRS-R - revised amytrophic lateral sclerosis functional rating scale, CSF - cerebrospinal fluid.}
\label{tab:unieatures_cox}
{\small\setlength{\tabcolsep}{4pt}
\begin{tabular}{|l|lr|lr|lr|lr|} 
\cline{2-9}
\multicolumn{1}{l|}{} & \multicolumn{2}{l|}{\multirow{2}{*}{Univariable (Significant Only)}} & \multicolumn{6}{c|}{Multivariable with Univariable Screening} \\ 
\cline{4-9}
\multicolumn{1}{l|}{} & \multicolumn{2}{l|}{} & \multicolumn{2}{l|}{Clinical} & \multicolumn{2}{l|}{Imaging} & \multicolumn{2}{l|}{Multimodal} \\ 
\hline
Variable & HR (95\% CI) & $p$ & HR (95\% CI) & $p$ & HR (95\% CI) & $p$ & HR (95\% CI) & $p$ \\ 
\hline
\multicolumn{1}{|l}{\textbf{Clinical}} &  & \multicolumn{1}{l}{} &  & \multicolumn{1}{l}{} &  & \multicolumn{1}{l}{} &  &  \\ 
\hline
ALSFRS-R & 0.70 (0.60--0.82) & \begin{tabular}[c]{@{}l@{}}$<$\textbf{\textbf{0.0001}}\\\end{tabular} & \textcolor[rgb]{0.2,0.2,0.2}{0.64 (0.53 -- 0.77)} & $<$\textbf{0.0001} & {\cellcolor[rgb]{0.753,0.753,0.753}} & {\cellcolor[rgb]{0.753,0.753,0.753}} & \textcolor[rgb]{0.2,0.2,0.2}{0.69 (0.55 -- 0.88)} & \textcolor[rgb]{0.2,0.2,0.2}{\textbf{0.0026}} \\ 
\hline
Diagnostic delay, mo & 0.77 (0.60--0.99) & \textbf{0.0409} & \textcolor[rgb]{0.2,0.2,0.2}{0.81 (0.61 -- 1.07)} & \textcolor[rgb]{0.2,0.2,0.2}{0.1385} & {\cellcolor[rgb]{0.753,0.753,0.753}} & {\cellcolor[rgb]{0.753,0.753,0.753}} & \textcolor[rgb]{0.2,0.2,0.2}{0.83 (0.61 -- 1.13)} & \textcolor[rgb]{0.2,0.2,0.2}{0.2346} \\ 
\hline
Age at diagnosis, yr & 1.48 (1.29--1.84) & \textbf{0.0005} & \textcolor[rgb]{0.2,0.2,0.2}{1.52 (1.21 -- 1.9)} & \textcolor[rgb]{0.2,0.2,0.2}{\textbf{0.0003}} & {\cellcolor[rgb]{0.753,0.753,0.753}} & {\cellcolor[rgb]{0.753,0.753,0.753}} & \textcolor[rgb]{0.2,0.2,0.2}{1.03 (0.74 -- 1.42)} & \textcolor[rgb]{0.2,0.2,0.2}{0.8646} \\ 
\hline
FTD &  &  &  &  & {\cellcolor[rgb]{0.753,0.753,0.753}} & {\cellcolor[rgb]{0.753,0.753,0.753}} &  &  \\
\hspace{5mm}No & 1.00, Ref & - & 1.00, Ref & - & {\cellcolor[rgb]{0.753,0.753,0.753}} & {\cellcolor[rgb]{0.753,0.753,0.753}} & 1.00, Ref & - \\
\hspace{5mm}Yes & 1.58 (1.04--2.41) & \textbf{0.0337} & \textcolor[rgb]{0.2,0.2,0.2}{1.42 (0.89 -- 2.26)} & \textcolor[rgb]{0.2,0.2,0.2}{0.1429} & {\cellcolor[rgb]{0.753,0.753,0.753}} & {\cellcolor[rgb]{0.753,0.753,0.753}} & \textcolor[rgb]{0.2,0.2,0.2}{1.17 (0.69 -- 2.01)} & \textcolor[rgb]{0.2,0.2,0.2}{0.5561} \\ 
\hline
MND Subtype &  &  &  &  & {\cellcolor[rgb]{0.753,0.753,0.753}} & {\cellcolor[rgb]{0.753,0.753,0.753}} &  &  \\
\hspace{5mm}Non-ALS & 1.00, Ref & - & 1.00, Ref & - & {\cellcolor[rgb]{0.753,0.753,0.753}} & {\cellcolor[rgb]{0.753,0.753,0.753}} & 1.00, Ref & - \\
\hspace{5mm}ALS & 2.40 (1.36--4.23) & \textbf{0.0026} & \textcolor[rgb]{0.2,0.2,0.2}{2.01 (1.09 -- 3.71)} & \textcolor[rgb]{0.2,0.2,0.2}{\textbf{0.0262}} & {\cellcolor[rgb]{0.753,0.753,0.753}} & {\cellcolor[rgb]{0.753,0.753,0.753}} & \textcolor[rgb]{0.2,0.2,0.2}{2.11 (1.03 -- 4.32)} & \textcolor[rgb]{0.2,0.2,0.2}{\textbf{0.0410}} \\ 
\hline
\multicolumn{1}{|l}{\textbf{Volumes}} &  & \multicolumn{1}{l}{} &  & \multicolumn{1}{l}{} &  & \multicolumn{1}{l}{} &  &  \\ 
\hline
Brain stem & \textcolor[rgb]{0.2,0.2,0.2}{0.66 (0.53 -- 0.83)} & \textcolor[rgb]{0.2,0.2,0.2}{\textbf{0.0003}} & {\cellcolor[rgb]{0.753,0.753,0.753}} & {\cellcolor[rgb]{0.753,0.753,0.753}} & \textcolor[rgb]{0.2,0.2,0.2}{0.67 (0.47 -- 0.94)} & \textcolor[rgb]{0.2,0.2,0.2}{\textbf{0.0218}} & \textcolor[rgb]{0.2,0.2,0.2}{0.74 (0.51 -- 1.06)} & \textcolor[rgb]{0.2,0.2,0.2}{0.1041} \\ 
\hline
CSF & \textcolor[rgb]{0.2,0.2,0.2}{1.40 (1.16 -- 1.70)} & \textcolor[rgb]{0.2,0.2,0.2}{\textbf{0.0005}} & {\cellcolor[rgb]{0.753,0.753,0.753}} & {\cellcolor[rgb]{0.753,0.753,0.753}} & \textcolor[rgb]{0.2,0.2,0.2}{1.33 (1.02 -- 1.73)} & \textcolor[rgb]{0.2,0.2,0.2}{\textbf{0.0332}} & \textcolor[rgb]{0.2,0.2,0.2}{1.24 (0.92 -- 1.68)} & \textcolor[rgb]{0.2,0.2,0.2}{0.1562} \\ 
\hline
Lateral ventricles & \textcolor[rgb]{0.2,0.2,0.2}{1.58 (1.32 -- 1.89)} & \textbf{$<$0.0001} & {\cellcolor[rgb]{0.753,0.753,0.753}} & {\cellcolor[rgb]{0.753,0.753,0.753}} & \textcolor[rgb]{0.2,0.2,0.2}{1.63 (1.14 -- 2.32)} & \textcolor[rgb]{0.2,0.2,0.2}{\textbf{0.0072}} & \textcolor[rgb]{0.2,0.2,0.2}{1.56 (1.03 -- 2.36)} & \textcolor[rgb]{0.2,0.2,0.2}{\textbf{0.0374}} \\ 
\hline
Hippocampus & \textcolor[rgb]{0.2,0.2,0.2}{0.59 (0.47 -- 0.73)} & \textbf{$<$0.0001} & {\cellcolor[rgb]{0.753,0.753,0.753}} & {\cellcolor[rgb]{0.753,0.753,0.753}} & \textcolor[rgb]{0.2,0.2,0.2}{1.10 (0.68 -- 1.78)} & \textcolor[rgb]{0.2,0.2,0.2}{0.7018} & \textcolor[rgb]{0.2,0.2,0.2}{1.38 (0.81 -- 2.37)} & \textcolor[rgb]{0.2,0.2,0.2}{0.2371} \\ 
\hline
Amygdala & \textcolor[rgb]{0.2,0.2,0.2}{0.59 (0.47 -- 0.73)} & \textbf{$<$0.0001} & {\cellcolor[rgb]{0.753,0.753,0.753}} & {\cellcolor[rgb]{0.753,0.753,0.753}} & \textcolor[rgb]{0.2,0.2,0.2}{0.65 (0.42 -- 0.99)} & \textcolor[rgb]{0.2,0.2,0.2}{\textbf{0.0473}} & \textcolor[rgb]{0.2,0.2,0.2}{0.62 (0.39 -- 0.97)} & \textcolor[rgb]{0.2,0.2,0.2}{\textbf{0.0381}} \\ 
\hline
Thalamus & \textcolor[rgb]{0.2,0.2,0.2}{0.78 (0.64 -- 0.96)} & \textcolor[rgb]{0.2,0.2,0.2}{\textbf{0.0173}} & {\cellcolor[rgb]{0.753,0.753,0.753}} & {\cellcolor[rgb]{0.753,0.753,0.753}} & \textcolor[rgb]{0.2,0.2,0.2}{1.74 (1.11 -- 2.71)} & \textcolor[rgb]{0.2,0.2,0.2}{\textbf{0.0149}} & \textcolor[rgb]{0.2,0.2,0.2}{1.38 (0.86 -- 2.23)} & \textcolor[rgb]{0.2,0.2,0.2}{0.1844} \\ 
\hline
Caudate & \textcolor[rgb]{0.2,0.2,0.2}{0.77 (0.64 -- 0.93)} & \textcolor[rgb]{0.2,0.2,0.2}{\textbf{0.0052}} & {\cellcolor[rgb]{0.753,0.753,0.753}} & {\cellcolor[rgb]{0.753,0.753,0.753}} & \textcolor[rgb]{0.2,0.2,0.2}{0.64 (0.43 -- 0.97)} & \textcolor[rgb]{0.2,0.2,0.2}{\textbf{0.0339}} & \textcolor[rgb]{0.2,0.2,0.2}{0.73 (0.46 -- 1.14)} & \textcolor[rgb]{0.2,0.2,0.2}{0.1609} \\ 
\hline
Putamen & \textcolor[rgb]{0.2,0.2,0.2}{0.72 (0.60 -- 0.87)} & \textcolor[rgb]{0.2,0.2,0.2}{\textbf{0.0007}} & {\cellcolor[rgb]{0.753,0.753,0.753}} & {\cellcolor[rgb]{0.753,0.753,0.753}} & \textcolor[rgb]{0.2,0.2,0.2}{1.55 (0.87 -- 2.73)} & \textcolor[rgb]{0.2,0.2,0.2}{0.1338} & \textcolor[rgb]{0.2,0.2,0.2}{1.19 (0.67 -- 2.11)} & \textcolor[rgb]{0.2,0.2,0.2}{0.5556} \\ 
\hline
Pallidum & \textcolor[rgb]{0.2,0.2,0.2}{0.75 (0.61 -- 0.91)} & \textcolor[rgb]{0.2,0.2,0.2}{\textbf{0.0041}} & {\cellcolor[rgb]{0.753,0.753,0.753}} & {\cellcolor[rgb]{0.753,0.753,0.753}} & \textcolor[rgb]{0.2,0.2,0.2}{0.77 (0.52 -- 1.13)} & \textcolor[rgb]{0.2,0.2,0.2}{0.1779} & \textcolor[rgb]{0.2,0.2,0.2}{0.84 (0.58 -- 1.22)} & \textcolor[rgb]{0.2,0.2,0.2}{0.363} \\ 
\hline
Cerebellum cortex & \textcolor[rgb]{0.2,0.2,0.2}{0.71 (0.57 -- 0.88)} & \textcolor[rgb]{0.2,0.2,0.2}{\textbf{0.0016}} & {\cellcolor[rgb]{0.753,0.753,0.753}} & {\cellcolor[rgb]{0.753,0.753,0.753}} & \textcolor[rgb]{0.2,0.2,0.2}{0.81 (0.58 -- 1.11)} & \textcolor[rgb]{0.2,0.2,0.2}{0.1900}\textcolor[rgb]{0.2,0.2,0.2}{} & \textcolor[rgb]{0.2,0.2,0.2}{0.86 (0.60 -- 1.23)} & \textcolor[rgb]{0.2,0.2,0.2}{0.4093} \\
\hline
\end{tabular}
}
\end{sidewaystable}



\begin{table}
\centering
\label{tab:coxfitmetrics}
\caption{Metrics assessing the Cox proportional hazards models' fits: c-index (concordance index) and AIC (Akaike Information Criterion).}
\begin{tabular}{|l|l|l|ll|} 
\cline{4-5}
\multicolumn{1}{l}{} & \multicolumn{1}{l}{} & & \multicolumn{2}{l|}{\textbf{Fit metrics} } \\ 
\hline
\textbf{Model} & \textbf{Factors Included} & \textbf{Number of Factors} & c-index & AIC \\ 
\hline
\multirow{2}{*}{Clinical} & All variables & 7 & 0.73 & 799.66 \\ 
\cline{2-5}
 & Univariable-screened & 5 & 0.73 & 796.40 \\ 
\hline
\multirow{2}{*}{Imaging} & All variables & 13 & 0.75 & 794.90 \\ 
\cline{2-5}
 & Univariable-screened & 10 & 0.74 & 799.60 \\ 
\hline
\multirow{2}{*}{Multimodal} & All variables & 20 & \textbf{0.78} & \textbf{787.53} \\ 
\cline{2-5}
 & Univariable-screened & 15 & 0.77 & 789.11 \\
\hline
\end{tabular}
\end{table}

Table~\ref{tab:allfeatures_cox} shows the HRs, confidence intervals, and significance $p$ values of the features in the univariable CPH, multivariable clinical CPH, multivariable imaging CPH, and multivariable multimodal CPH.
The significantly harmful univariable factors are an older age of diagnosis (HR=1.48), co-presence of FTD (HR=1.58), having ALS MND (HR=2.40), and larger volumes in CSF (HR=1.40) and lateral ventricles (HR=1.58).
Significantly protective factors include higher baseline ALSFRS-R (HR=0.70), a longer diagnostic delay (0.77), and larger volumes in the brain stem (HR=0.66), hippocampus (HR=0.59), amygdala (HR=0.59), thalamus (HR=0.78), caudate (HR=0.77), putamen (HR=0.72), pallidum (HR=0.75) and cerebellum cortex (HR=0.71).
The only features included that did not significantly affect survival were sex, bulbar site of onset, cerebral white matter volume, cerebellum white matter volume, and cerebral cortex volume.


\subsection{Multivariable}

\subsubsection{Clinical}
When the clinical features were input into a multivariable CPH, baseline ALSFRS-R was a significant protective factor (HR=0.62), and ALS MND and older age at diagnosis were significantly harmful (HRs of 1.96 and 1.53). 
Diagnostic delay and co-presence of FTD were no longer significant in the multivariable CPH.

The proportional hazards assumptions were broken by age at diagnosis ($p=0.002$) and diagnostic delay ($p=0.038$).
In an effort to correct the broken assumptions, another multivariable CPH was fit with only the univariably-significant clinical factors.
Table~\ref{tab:unieatures_cox} shows the results from the univariable-screened CPH models.
This fixed the broken proportional hazards assumptions and the same factors remained significant: high baseline ALSFRS-R (HR=0.64), age at diagnosis (HR=1.52), and ALS MND (HR=2.01).


\subsubsection{Imaging}
The multivariable imaging CPH resulted in three significant survival factors: lateral ventricles (harmful, HR=1.52), cerebral white matter (harmful, HR=2.35), and caudate (protective, HR=0.62).
However, the cerebral white matter and CSF variables broke the CPH assumptions ($p=0.0325$ and $0.005$ respectively).

When only the univariably-significant imaging factors are input into a multivariable CPH, more factors were significantly associated with survival, shown in Table~\ref{tab:unieatures_cox}.
Higher volumes of the brain stem (HR=0.67), amygdala (HR=0.65), and caudate (HR=0.64) were protective, and higher volumes of the CSF (HR=1.33), lateral ventricles (HR=1.63), and thalamus (HR=1.74) were harmful.

\subsubsection{Multimodal}
Only three factors were significant in the multimodal multivariable CPH: baseline ALSFRS-R (HR=0.68), ALS MND (HR=2.32), and cerebral white matter volume (HR=2.43).
The proportional hazards assumption was broken by sex ($p=0.0069$), bulbar site of onset ($p=0.0298$), and cerebral white matter ($p=0.0497$).

Screening input variables by their univariable significance resulted in a CPH that had no broken assumptions and four significant survival factors: two clinical and two imaging.
Higher baseline ALSFRS-R (HR=0.69) and larger amygdala volume (HR=0.62) were significantly protective, and ALS MND (HR=2.11) and larger lateral ventricle volume (HR=1.56) were significantly harmful.

Table~\ref{tab:coxfitmetrics} shows the metrics of model fit for the multivariable CPH models.
The multimodal multivariable models resulted in the best fit metrics, both when considering all the models and also when considering only the univariable-screeened models and the ``all variable" models.

\section{Discussion}


Logic of significant factors

Bulbar onset not part of it 

Limitations of univariable

\begin{itemize}
    \item Discuss the logic of the significant features
    \item bulbar onset not a signficant factor which is surprising, but could be to do with  a fairly high proportion of our cohort having bulbar onset, and maybe testing for respiratory onset would be better
    \item What are the limitations of a univariable model - doesn't take into account the relationships between features
\end{itemize}

Clinical model:



\begin{itemize}
    \item Proportional hazards were broken by diagnosis age and diagnostic delay
    \item Same features significant with univariable screening
    \item Why are we ok with this? Because we have a relatively small sample size, stratifying the model may lead to worse results than just including the features and acknowledging the violation
    \item And the hazard directions are consistent with the literature, so the model isn't completely wrong
    \item Hazard ratios make sense
    \item FTD and diagnostic delay lost significance - could be that the other factors that were significant account for the survival risk from these factors. For example, FTD and age at diagnosis are positively correlated in our cohort, so perhaps only one was needed to be kept to keep the effect on survival. Likewise, ALS MND and diagnostic delay are negatively correlated, which could explain why harmful ALS MND remained significant but protective diagnostic delay lost significance.
\end{itemize}

Imaging model:
\begin{itemize}
    \item Significant features include cerebral white matter which was not significant in the unvariable model
    \item With the univariable model, more features were significant, and all were clinically intuitive apart from thalamus
    \item Protective in univariable, hazard in multivariable - why? could be to do with the high collinearity still in the model
\end{itemize}

Clinical and imaging model:
\begin{itemize}
    \item proportional hazards fixed with uni significant features and resulting features are all clinically intuitive, although amygdala being significant over more intuitive features is surprising
    \item Two clinical and two imaging features were significant - shows that the imaging features are adding information to the model
    \item Talk about each significant feature
    \item Fit statistics: fit improved with multimodal features - this could be just because we're adding more features, but it could also be because we're adding more information
\end{itemize}

Limitations:
\begin{itemize}
    \item Small-ish sample size
    \item Multi-site without harmonisation
    \item Univariable screening - a controversial approach.
    \item However, colinearity of features doesn't diminish the results
\end{itemize}

Take home message:
\begin{itemize}
    \item Using imaging-derived feature with clinical features increases concordance, more information into the model is better, makes the clinical measure significant as well when it wasn't before
    \item Would this result also be sustained when using more sophisticated machine learning models? Instinct says yes because they can handle more features and more complex relationships between features
\end{itemize}

\section{Conclusion}
\begin{itemize}
    \item Link to next chapter: we've shown that using imaging and clinical together is better than using them alone
    \item Let's see if we can improve on this by using more sophisticated machine learning models
\end{itemize}