\chapter{Fusilli Development}
\label{fusilli_development}

\section{Introduction}
\begin{itemize}
    \item Motivation behind developing Fusilli
\end{itemize}

\section{Methods}
\begin{itemize}
    \item Finding models to put in Fusilli
    \item Categorising models: Cui et al. paper
    \item Data types: tabular-tabular and tabular-image
    \begin{itemize}
        \item Useful for clinical+brain volumes and clinical+whole MRI
    \end{itemize}
    \item Machine learning tasks: regression and classification
    \item Model modifications: important for aligning models best with different data e.g. making models more complex for higher res MRI
\end{itemize}

\section{Results}
\begin{itemize}
    \item Table of models
    \item JOSS paper
\end{itemize}

\section{Discussion}
\begin{itemize}
    \item Successes
    \item Limitations
\end{itemize}

\section{Conclusion}
\begin{itemize}
    \item Link to next chapter:
    \begin{itemize}
        \item We've got our models
        \item Shall we see which is the most useful for predicting survival in MND?
    \end{itemize}
\end{itemize}
