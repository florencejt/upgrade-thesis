\chapter{Fusilli: Developing a Data Fusion Python Library}
\label{fusilli_development}

\textbf{Linking sentence from Cox chapter}: 
We've seen how multimodal data affects a survival Cox model in MND.
What about machine learning for multimodal data?
This chapter discusses multimodal data fusion in more detail and describe the development of Fusilli, a Python package for multimodal data fusion experimentation and analysis.

\section{Introduction}

Multimodal data fusion is the process of combining data from different sources to make predictions or decisions.
The goal of combining different modalities is to improve the performance of a model by leveraging the relevant information from each modality and fusing them in a way that improves the model's performance.
The application areas of multimodal data fusion are wide-reaching, including agriculture, disaster management, robotics, healthcare, and many others.


Describing multimodal data fusion:
\begin{itemize}
    \item This is an area of research called multimodal data fusion but it can be called more names than that: multi-view, cross-heterogeneous, etc
    \item The application areas of multimodal data fusion are very wide reaching: agriculture, disaster management, robotics, healthcare etc
    \item Additionally, the types of models used in multimodal data fusion can vary a lot, from geometric deep learning to relatively simple neural network architectures.
\end{itemize}

What did I want to do with data fusion:
\begin{itemize}
    \item Multimodal data fusion is such a large field with many applications
    \item We saw in the lit review sections on machine learning that deep learning has not been done much with MND, and deep-learning data fusion has been done once by van der Burgh
    \item and deep learning data fusion has not been done with tabular and image data in MND
    \item For my PhD's overall goal of applying multimodal data fusion to motor neuron disease, it is important to have an understanding of the different models and their performances
    \item This would require a large amount of experimentation and analysis, which would be difficult to do manually
    \item So I wanted to compare different fusion methods to see what worked best
\end{itemize}

Issues I came across when starting to look at models:
\begin{itemize}
    \item Difficulty finding models in the first place: called different names
    \item As I kept looking, I kept finding lots of and lots of models with different architectures   
    \item Not all papers include code
    \item Code is not standardised - written in different languages, varying availability, varying guidance
    \item There are some collections/repos but not what I was looking for
\end{itemize}

Aims of this work:
\begin{itemize}
    \item To create a bank of models that can be used for multimodal data fusion
    \item To create a standardised way of comparing models
    \item Make it open source for others to use who are in similar positions to me
\end{itemize}

Chapter aims:
\begin{itemize}
    \item To describe the development of Fusilli
    \item To describe the design choices made in developing Fusilli
    \item To describe the implemented models in Fusilli
    \item To describe the testing of Fusilli
    \item To describe the reception of Fusilli
\end{itemize}

\section{Fusion Methods}

\subsection{Different types of models}

Talk about the review paper and the categories

\begin{itemize}
    \item Gathered models and then categorised them according to Cui et al. paper - a paper about diagnosis and prognosis deep learning multi modal fusion models.
    \item The paper included many models with different architectures and categorised based off underlying architecture.
    \item Usually categorisations are broad with early, late, and intermediate fusion but this paper had more categories which was useful for me.
    \item \textbf{Categories figure}
    \item Some models can fall into multiple categories, so chose the category that best described the model.
\end{itemize}


\subsection{Finding the models}
Literature search - what words I used, not just healthcare but it came up a lot 

\subsection{Models included}

Table of models, and benchmarks (uni-modal)

Link to documentation with diagrams of models

\section{Software Design}

\subsection{Design Goals}
List in bullet points the design goals of fusilli and how they were met: 
\begin{enumerate}
    \item Modularity: Easily add models in the future and easy adjustments
    \item Beginner-friendly: default parameters for people who don't know how to do deep learning coding, also extensive documentation
    \item Expert-friendly: for people who are more familiar with deep learning, they should be able to change the training parameters, modify the models, access trained models for further experiments
    \item Wide-ranging: include a wide variety of models
    \item Widely-applicable: all models to be used for different tasks, all image dimensions and tabular data
    \item Open-source: want people to contribute with their own models. To make sure it doesn't break when people do this, it needs testing. 
\end{enumerate}


\section{Results}

\subsection{Diagram of workflow}

\subsection{Example usage and outputs}
Quick-start script with comments going through each line

Figures with output figures from the example notebooks

\subsection{Documentation}
Examples, templates for contributing, etc


\subsection{Reception}
\begin{itemize}
    \item JOSS paper - under review 
    \item GitHub stars and forks and articles
\end{itemize}

\section{Discussion}
\begin{itemize}
    \item Implemented a wide variety of models
    \item It will help people to know if data fusion is useful for their task and, if so, which model is best for their task
    \item Limitation and future work: Data inputs - images have to be .pt files and tabular data has to be .csv files. Would be nice to extend data inputs to be jpegs or niis for people less comfortable with Python.
    \item Limitation and future work: Only 2 modalities - could be extended to more
\end{itemize}

\section{Conclusion}
TBC...
% \begin{itemize}
%     \item Fusilli is a Python package for multimodal data fusion experimentation and analysis, specifically tackling the problem of lack of ways to compare models and lack of standardisation in the field.
%     \item I specifically made it for my PhD in fusing different data modalities in MND prognosis prediction
%     \item Link to next chapter on PPMI, ADNI, MIMIC: data fusion can be applied to any task where there are multiple data modalities describing one thing
%     \item Before going into MND, is there a clear benefit to different types of multimodal data fusion? Is there a consensus on the best approaches for different tasks?
%     \item MND sample size is small so we want to make sure we're using the best models by testing them on larger datasets with different applications
% \end{itemize}
