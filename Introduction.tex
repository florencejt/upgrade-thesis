\chapter{Introduction}
\label{introduction}

%Summary of this thesis first?
%
%Motor neuron disease (MND) is a rare and fatal neurodegenerative disease of the upper (UMN) and lotor motor neurons (LMN).
%MND is a clinically heterogeneous disease.
%There are multiple subtypes, all of which have varying survival times and clinical presentations.
%The most common of these is amyotrophic lateral sclerosis (ALS), which accounts for THIS MANY \% of patients.
%The less common subtypes are Progressive Spinal Muscular Atrophy (PMA), which is characterised by exclusively LMN involvement, Primary Lateral Sclerosis (PLS), which is exclusively UMN involvement, and Progressive Bulbar SOMETHING (PBP).
%Distinguishing between the various subtypes of MND during diagnosis is important because they have different prognoses and presentations.
%For example, PLS has a more benign prognosis than ALS because there is reduced respiratory involvement, and patients have a chance of living to a normal lifespan~\cite{statlandPrimaryLateralSclerosis2015}.
%There is currently no cure for ALS and survival time from diagnosis is usually between 3 to 4 years~\cite{swinnenPhenotypicVariabilityAmyotrophic2014, goutmanRecentAdvancesDiagnosis2022a}, with 10\% of patients living more than 10 years~\cite{pupilloLongtermSurvivalAmyotrophic2014}.
%Rates of ALS within a population are affected by both ancestry and sex.
%Global ALS incidence is estimated to be 1.68 per 100,000, and European populations experience higher incidence of 1.71 to 1.89 per 100,000~\cite{marinVariationWorldwideIncidence2017}.
%Additionally, ALS affects 1.3 men for every woman~\cite{fontanaTimetrendEvolutionDeterminants2021}.

Motor neuron disease (MND) is a fatal neurodegenerative condition affecting both upper (UMN) and lower motor neurons (LMN).
It encompasses various subtypes, notably amyotrophic lateral sclerosis (ALS), which constitutes 90\% of cases~\cite{filippiMotorNeuronDiseases2015}.
%Other subtypes include Progressive Spinal Muscular Atrophy (PMA), Primary Lateral Sclerosis (PLS), and Progressive Bulbar Palsy (PBP), characterised by varying degrees of UMN and LMN involvement.
%Distinguishing between these subtypes during diagnosis is crucial due to their distinct prognoses and clinical presentations.
%For instance, PLS tends to have a more favourable prognosis compared to ALS due to less respiratory involvement, offering the possibility of a normal lifespan~\cite{statlandPrimaryLateralSclerosis2015}.
There is currently no cure for ALS, with an average survival time of 2 to 4 years post-diagnosis, although a minority of patients survive beyond 10 years~\cite{swinnenPhenotypicVariabilityAmyotrophic2014,goutmanRecentAdvancesDiagnosis2022a,pupilloLongtermSurvivalAmyotrophic2014}.
Incidence rates of ALS vary globally, with European populations and males experiencing higher rates~\cite{marinVariationWorldwideIncidence2017,fontanaTimetrendEvolutionDeterminants2021}.


It is unknown exactly what causes MND, but genetics play a large role.
Historically, ALS has been categorised into familial ALS (fALS) and sporadic ALS (sALS), distinguished by whether there is a family history of ALS.
However, with 10\% of sALS patients exhibiting fALS-associated mutations and a genetic contribution estimated at 61\%, there is a blurred boundary between sALS and fALS causes.
%, prompting a reclassification into ``genetically confirmed`` and ``non-genetically confirmed`` ALS~\cite{hanbyRiskRelativesPatients2011, al-chalabiEstimateAmyotrophicLateral2010, feldmanAmyotrophicLateralSclerosis2022}.
%However, 10\% of sALS patients have mutations associated with fALS~\cite{hanbyRiskRelativesPatients2011}, and genetic contribution of sALS has been estimated as 61\% \cite{al-chalabiEstimateAmyotrophicLateral2010}.
%This hazy border between what differentiates the causes of sALS and fALS has led to the movement of recategorising ALS into ``genetically confirmed`` and ``non-genetically confirmed``~\cite{feldmanAmyotrophicLateralSclerosis2022}.
As of 2022, there have been over 40 genes associated with ALS~\cite{goutmanRecentAdvancesDiagnosis2022a}, the most common of which is the C9orf72 hexanucleotide repeat expansion, which occurs in 5--15\% of fALS cases~\cite{vanesAmyotrophicLateralSclerosis2017}.
%C9orf72 is not unique to ALS; it is also linked to increased risks of frontotemporal dementia, Parkinson's Disease, Huntingdon's Disease, Alzheimer's Disease, Schizophrenia, and bipolar disorder.
Apart from genetic risks, there are investigations into environmental risk factors for ALS. A commonly studied factor is intense physical exercise, including involvement in professional sports and military service~\cite{mckayMilitaryServiceRelated2021, lacortePhysicalActivityPhysical2016}.
%, and there are multiple theories as to why this might be involvement in increased risk of developing ALS. ADD MORE HERE.


The primary symptom of MND is progressive motor loss in voluntary muscles~\cite{vanesAmyotrophicLateralSclerosis2017}.
%, sparing involuntary movements like pupillary responses~\cite{vanesAmyotrophicLateralSclerosis2017}.
This progressive motor loss can manifest as reduced limb movement, difficulty swallowing (dysphagia), difficulty speaking (dysarthria), and impaired respiratory function, which is usually the cause of death.
At symptom onset, this motor loss is usually focused on one body segment, and the weakness spreads in a predictive pattern to the contralateral side in 85\% of patients~\cite{walhoutPatternsSymptomDevelopment2018}.
Ongoing research delves into the incomplete understanding of the neurodegenerative mechanism, particularly focusing on TDP-43 aggregation in the brain, a common feature in nearly all ALS cases~\cite{blokhuisProteinAggregationAmyotrophic2013}.

Recently, there's been heightened attention on the cognitive and behavioural aspects of MND, affecting 35 to 50\% of ALS patients, with factors such as genetics and symptom onset site influencing their occurrence~\cite{yangRiskFactorsCognitive2021, chioALSPhenotypeInfluenced2020}.
%Recently, there has been increased focus on the non-motor symptoms of MND: cognitive and behavioural changes.
%These changes are recognised to occur in 35 to 50\% of ALS patients, and some patient characteristics such as genetic factors and site of symptom onset affect the likelihood of cognitive involvement~\cite{yangRiskFactorsCognitive2021, chioALSPhenotypeInfluenced2020}.
FTD occurs in 15\% of ALS cases, and ALS-FTD is considered a spectrum because of the reciprocal links observed: 12.5\% of behavioural-variant FTD patients progress to ALS~\cite{strongAmyotrophicLateralSclerosis2017}.
The most common behavioural symptoms affecting 10\% of ALS patients are apathy and loss of sympathy~\cite{abrahamsScreeningCognitionBehaviour2014}.
Other common symptoms are issues with language fluency, social cognition, and executive function~\cite{beeldmanCognitiveProfileALS2016}, although long-term and spatial memory are usually spared~\cite{crockfordALSspecificCognitiveBehavior2018}.
Although these cognitive and behavioural symptoms are not the cause of death of MND patients, increased changes may signal faster disease progression and increased caregiver burden, so it is important to recognise and measure these changes clinically.

There is no definitive diagnostic test for MND, and each patient undergoes a tailored investigation of differential diagnosis.
%The diagnosis process can also be tricky because upper and lower motor neuron involvement may not happen simultaneously.
%Additionally, there are a number of ALS-mimic diseases, which have similar symptoms to ALS but more benign prognoses, such as SPINAL ATROPHY?.
One diagnostic aid is the El Escorial criteria, which categorises patients based on the spread of progressive weakness~\cite{ludolphRevisionEscorialCriteria2015}.
%A diagnostic criteria was developed in GIVE THE DATE, called the El Escorial criteria (find a citation for this), which is used for patients with a history of progressive weakness that has spread.
%Patients are categorised as possible, probable, probable (laboratory supported), or definite ALS.
Diagnostic assessment currently does not include cognitive or behavioural changes, but tools are being developed and more commonly used to monitor these changes, such as Edinburgh Cognitive and Behavioural ALS Screen~\cite{abrahamsScreeningCognitionBehaviour2014}.
Due to the lengthy diagnostic process, general unawareness of MND symptoms, and varied speeds of symptom progression, the delay between symptom onset and diagnosis is on average 12 months, roughly halfway through the disease pathway~\cite{mitchellTimelinesDiagnosticEvaluation2010}.

Disease progression is assessed using the ALSFRS-R (revised amyotrophic lateral sclerosis functional rating scale), a questionnaire evaluating patients' ability to perform disease-related functions, with scores ranging from 4 (normal function) to 0 (loss of function), yielding a total score ranging from 48 to 0~\cite{cedarbaumALSFRSRRevisedALS1999}.
While often employed as the primary outcome measure in clinical trials, there is debate over its appropriateness due to its broad domain coverage, prompting suggestions to focus on domain-specific scores like limb movement or bulbar function~\cite{vaneijkOldFriendWho2021, rooneyWhatDoesALSFRSR2017}.
Additionally, MND patients may be staged using systems such as the King's Clinical Staging~\cite{rocheProposedStagingSystem2012} or the ALS Milano-Torino Staging~\cite{chioDevelopmentEvaluationClinical2015}, though their widespread clinical adoption remains limited~\cite{feldmanAmyotrophicLateralSclerosis2022, fangComparisonKingMiToS2017}.

Common therapies include gastrostomy, weight maintenance, and non-invasive ventilation (NIV)~\cite{bourkeEffectsNoninvasiveVentilation2006}.
In the UK, Riluzole is the primary treatment, potentially offering benefits in advanced stages, with a demonstrated increase in median survival by 3 months~\cite{millerRiluzoleAmyotrophicLateral2012, hinchcliffeRiluzoleRealworldEvidence2017}.
%Outside the UK, Edaravone shows promise in slowing disease progression but faces scrutiny regarding trial criteria and safety~\cite{witzelSafetyEffectivenessLongterm2022}.


Imaging, such as brain magnetic resonance imaging (MRI) and spinal MRI, is currently used in the diagnostic pathway to rule out mimicking diseases.
Although not used for prognosis clinically, neuroimaging-derived measures have been shown to be associated with patient survival and progression~\cite{mullerLargescaleMulticentreCerebral2016,agostaMRIPredictorsLongterm2010}.
In other diseases, machine learning (ML) has had great success in predicting prognosis from medical imaging data, including brain MRI~\cite{gerratyMachineLearningParkinson2023}.
However, in MND, most ML studies do not use imaging data, and more work has been conducted on using ML with clinical data for prognosis, with some success~\cite{grollemundMachineLearningAmyotrophic2019, papaizMachineLearningSolutions2022}.
There is a gap in the literature for using imaging data for MND prognosis prediction, and for combining imaging and clinical data for this purpose.

The work presented in this thesis explores the creation of a machine learning model for MND prognosis prediction using clinical and imaging data from the diagnostic appointment.
The combination of clinical and imaging data is known as \textit{multimodal data fusion}, and is a growing area of machine learning research in the medical imaging field and beyond~\cite{cuiDeepMultimodalFusion2022, mohsenArtificialIntelligencebasedMethods2022}.

\section{Motivation}


Prognosis is the prediction of the course of a disease and is important for both patients and clinicians.
For patients, an accurate prognosis can help them plan their lives and feel empowered to make decisions about their care~\cite{talbotClinicalToolPredicting2016, vaneenennaamDiscussingPersonalizedPrognosis2021}.
For clinical use, prognosis can be used to stratify patients for clinical trials to investigate how different treatments affect different patient groups~\cite{berryImprovedStratificationALS2018}.
Additionally, a predicted disease course could be used as an outcome measure for clinical trials~\cite{kiernanImprovingClinicalTrial2021}.

Imaging from the diagnostic appointment is a potentially valuable source of data for prognosis prediction.
Applying imaging for this purpose would not require any extra data collection, as it is already being taken for differential diagnosis.
Moreover, a baseline scan is easier for patients to undergo than a scan once their disease has progressed, due to the physical limitations of progressed MND .

% Combingin with clinical data is a good idea because there is already good evidence of clinical prognostic factors, and imaging could add to this in a data driven way
Since MND is multifactorial and complex, investigating the value of all of the available data collected clinically for prognosis is a logical step.
Clinical prognostic factors for MND are well established~\cite{suPredictorsSurvivalPatients2021}, but prognostic factors derived from imaging data are less well understood~\cite{bedeLessonsALSImaging2014}.
Combining these known clinical factors with imaging data could provide a more accurate prognosis than using clinical or imaging factors alone, and could reveal or corroborate new prognostic factors that are not yet known.
ML techniques, especially deep learning, are well suited to this task because they are designed to find complex patterns in large datasets that traditional statistical methods may miss.

%\begin{itemize}
%    \item Why MND prognosis? Patients want it, good for clinical trial stratification, good for planning treatment
%    \item Why imaging? Already being taken at diagnosis for differential diagnosis, might as well use it for prognosis - healthcare economics, no extra data collection. Also has shown promise in the literature. A lot fo machine learnign literature on imaging too - good for deep learning because it's so big and complex.
%    \item Baseline data because difficult for patients to be scanned once their disease has progressed
%    \item Why data fusion? MND is multifactorial, so it makes sense to use all the data available to us. Also, it's a growing area of machine learning, and has shown promise in other diseases.
%\end{itemize}

\section{Project Aims}

The aims of this project are as follows:

\begin{enumerate}[label=Aim \arabic*:]
    \item To identify factors that are associated with MND prognosis and assess the value of imaging in survival analysis.
    \item To search the literature for multimodal data fusion methods and compare their performance on MND prognosis prediction with clinical and imaging-derived features.
    \item To investigate the added value of different imaging modalities and preprocessing methods, such as sub-region segmentation, texture analysis, and whole brain analysis, in multimodal data fusion for MND prognosis prediction.
    \item To develop the optimal multimodal data fusion model further by adding the capability to include more data modalities, such as fluid biomarkers and natural language processing (NLP) from radiological reports.
    \item To assess how the performance of the proposed prognostic model compares to existing models and clinical prognostic factors, and to assess its suitability to different prognostic tasks, such as stratifying patients for clinical trials or predicting treatment needs.
\end{enumerate}


\section{Upgrade Thesis Outline}

Chapter~\ref{literature_review} provides a literature review of multimodal prognostic factors in MND, machine learning for MND prognosis, and multimodal methods applied to MND.
Chapter~\ref{cox_proportional_hazards_model} presents the results of Aim 1, identifying survival factors in our dataset through Cox proportional hazards models.
Chapter~\ref{fusilli_development} describes multimodal data fusion in more detail and outlines the development and design of a Python package, Fusilli, for training, evaluating, and comparing multimodal data fusion methods, and presents the results of Aim 2.
Chapter~\ref{fusilli_on_mnd} presents the results of Aim 3, applying Fusilli to MND prognosis prediction with clinical and imaging data.
Finally, Chapter~\ref{conclusions_and_future_work} summarises the findings of this thesis and outlines future work.