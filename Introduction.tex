\chapter{Introduction}
\label{introduction}

\textbf{Motor neuron disease}

\begin{itemize}
    \item General description of MND: fatal neurodegenerative condition
    \item Survival time
    \item Incidence
    \item Types of MND: ALS, PLS, PMA, etc
    \item Causes
    \item Symptoms: motor, cognitive and behavioural
    \item Diagnostic delay
    \item Path to diagnosis
    \item Treatments
    \item Prognosis options
\end{itemize}

\textbf{Imaging in MND}
\begin{itemize}
    \item Only used in the clinical pathway for diagnosis: Ruling out ALS mimics, El Escorial scale, imaging to rule out other conditions
    \item Neuroimaging-derived measures have been shown to be associated with patient prognosis - insert citations here
    \item Machine learning for images, and specifically medical imaging, has been shown to be useful for prognosis in other diseases, and to some extent for MND
    \item Imaging can be difficult in MND later in the disease due to muscle atrophy and difficulty in positioning as well as swallowing and saliva problems
    \item A growing area of machine learning is multimodal data fusion, where different types of data are combined to improve the performance of the model.
    \item Imaging has not been included yet into the clinical pathway for prognosis, but perhaps it has a role to play when combined with clinical data through multimodal data fusion AI
\end{itemize}

\textbf{This work}
This work aims to create a prognostic tool for MND using multimodal data fusion AI, combining clinical, imaging and other data types together at the diagnostic appointment


\section{Motivation}

\begin{itemize}
    \item Why MND prognosis? Patients want it, good for clinical trial stratification, good for planning treatment
    \item Why imaging? Already being taken at diagnosis for differential diagnosis, might as well use it for prognosis - healthcare economics, no extra data collection. Also has shown promise in the literature. A lot fo machine learnign literature on imaging too - good for deep learning because it's so big and complex.
    \item Why data fusion? MND is multifactorial, so it makes sense to use all the data available to us. Also, it's a growing area of machine learning, and has shown promise in other diseases.
\end{itemize}

\section{Project Aims}

\begin{enumerate}
    \item Aim 1: investigate value of imaging in non-ML data fusion prognosis
    \item Aim 2: explore methods for multimodal data fusion in the literature and apply to larger multimodal neurodegenerative disease datasets
    \item Aim 3: apply to MND prognosis prediction with clinical and imaging data: find the value of imaging and the best way to use it (whole brain, regions of interest, texture analysis, etc
    \item Aim 4: Add more modalities such as fluid biomarkers and NLP from radiological reports
    \item Aim 5: Create the optimal multimodal data fusion model for MND prognosis and assess added values of different modalities
\end{enumerate}

\section{Upgrade Thesis Outline}