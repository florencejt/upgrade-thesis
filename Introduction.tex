\chapter{Introduction}
\label{introduction}

Summary of this thesis first?

Motor neuron disease (MND) is a rare, fatal, neurodegenerative disease of the upper (UMN) and lotor motor neurons (LMN).

MND is a clinically heterogeneous disease. There are multiple subtypes, all of which have varying survival times and clinical presentations. The most common of these is amyotrophic lateral sclerosis (ALS), which accounts for THIS MANY \% of patients. The less common subtypes are Progressive Spinal Muscular Atrophy (PMA), which is characterised by exclusively LMN involvement, Primary Lateral Sclerosis (PLS), which is exclusively UMN involvement, and Progressive Bulbar SOMETHING (PBP). Distinguishing between the various subtypes of MND during diagnosis is important because they have different prognoses and presentations. For example, PLS has a more benign prognosis than ALS because there is less respiratory involvement, and patients have a chance of living to a normal lifespan \cite{statlandPrimaryLateralSclerosis2015}.

There is currently no cure for ALS and survival time from diagnosis is usually between 3 to 4 years \cite{swinnenPhenotypicVariabilityAmyotrophic2014, goutmanRecentAdvancesDiagnosis2022a}, with 10\% of patients living more than 10 years \cite{pupilloLongtermSurvivalAmyotrophic2014}.

Rates of ALS within a population are affected by both ancestry and sex. Global ALS incidence is estimated to be 1.68 per 100,000, and European populations experience higher incidence of 1.71 to 1.89 per 100,000 \cite{marinVariationWorldwideIncidence2017}. Additionally, ALS affects 1.3 men for every woman \cite{fontanaTimetrendEvolutionDeterminants2021}.

It is unknown exactly what causes MND, but genetics play a large role. Historically, ALS has been categorised into familial ALS (fALS), meaning there is a family history of ALS, and sporadic ALS (sALS), meaning no family history. However, 10\% of patients with sALS have mutations associated with fALS \cite{hanbyRiskRelativesPatients2011}, and genetic contribution of sALS has been estimated as 61\% \cite{al-chalabiEstimateAmyotrophicLateral2010}. This hazy border between what differentiates the causes of sALS and fALS has led to the movement of recategorising ALS into ``genetically confirmed`` and ``non-genetically confirmed`` \cite{feldmanAmyotrophicLateralSclerosis2022}.

As of 2022, there have been over 40 genes associated with ALS (CITE GOUTMAN 2022B). The most common of these is the C9orf72 hexanucleotide repeat expansion, which occurs in 5-15\% of fALS patients \cite{vanesAmyotrophicLateralSclerosis2017}. C9orf72 is not unique to ALS; it is also linked to increased risks of frontotemporal dementia, Parkinson's Disease, Huntingdon's Disease, Alzheimer's Disease, Schizophrenia, and bipolar disorder.
Apart from genetic risks, there is investigation into environmental risk factors for ALS. A commonly-studied factor is intense physical exercise, including involvement in professional sports and military service \cite{mckayMilitaryServiceRelated2021, lacortePhysicalActivityPhysical2016}, and there are multiple theories as to why this might be involvement in increased risk of developing ALS. ADD MORE HERE.

The main symptom of MND is progressive motor loss in any voluntary muscle, which means that any involuntary movement such as pupillary movement is unsually unaffected \cite{vanesAmyotrophicLateralSclerosis2017}. This progressive motor loss can manifest as reduced limb movement, difficulty swallowing (dysphagia), difficulty speaking (dysarthria), and impaired respiratory function, which is usually the cause of death for MND patients. At symptom onset, this motor loss is usually focused on one body segment, and the weakness spreads in a predictive pattern to the contralateral side in 85\% of patients \cite{walhoutPatternsSymptomDevelopment2018}. The exact mechanism driving this neurodegeneration is not completely understood, and both cellular and molecular processes are being investigated. A specific process "under the microscope" is the aggregation of TDP-43 in the brain, which has been observed in nearly all ALS patients \cite{blokhuisProteinAggregationAmyotrophic2013}.

Recently, there has been increased focus on the non-motor symptoms of MND: cognitive and behavioural changes. These changes are recognised to occur in 35 to 50\% of ALS patients, and some patient characteristics such as genetic factors and site of symptom onset affect the likelihood of cognitive involvement \cite{yangRiskFactorsCognitive2021, chioALSPhenotypeInfluenced2020}. Frontotemporal dementia (FTD) occurs in 15\% of ALS patients, and ALS-FTD is often described as a spectrum \cite{strongAmyotrophicLateralSclerosis2017}. The spectrum also goes two ways, with 12.5\% of behavioural-variant FTD patients going on to develop ALS.

The most common behavioural symptoms affecting 10\% of ALS patients are apathy and loss of sympathy \cite{abrahamsScreeningCognitionBehaviour2014}. Other common symptoms are issues with language fluency, social cognitioin, and executive function \cite{beeldmanCognitiveProfileALS2016}. Long-term memory and spatial memory are usually spared \cite{crockfordALSspecificCognitiveBehavior2018}. Although these cognitive and behavioural symptoms are not the cause of death of MND patients, increased changes may signal faster disease progression and increased caregiver burden, so it is important to recognise and measure these changes clinically.

There is no definitive diagnostic test for MND, and each patient undergoes a tailored investigation of differential diagnosis. The diagnosis process can also be tricky because upper and lower motor neuron involvement may not happen simultaneously. Additionally, there are a number of "ALS mimic diseases", which have similar symptoms to ALS but more benign prognoses, such as SPINAL ATROPHY?. A diagnostic criteria was developed in GIVE THE DATE, called the El Escorial criteria (find a citation for this), which is used for patients with a history of progressive weakness that has spread. Patients are categorised as possible, probable, probable (laboratory supported), or definite ALS.
The diagnostic assessment currently does not include cognitive or behavioural changes, but tools are being developed and more commonly used in order to monitor these changes, such as ECAS \cite{abrahamsScreeningCognitionBehaviour2014}.

Due to the length diagnostic process, general unawareness of MND symptoms, and varied speeds of symptom progression, the delay between symptom onset and diagnosis is on average 12 months and roughly halfway through the disease pathway \cite{mitchellTimelinesDiagnosticEvaluation2010}.

Disease progression is clinically monitored using the ALSFRS-R (the revised amytrophic lateral sclerosis functional rating scale) \cite{cedarbaumALSFRSRRevisedALS1999}. This is a questionnaire through which patients are scored on their ability to complete disease-related functions, such as swallowing or rolling over in bed. The scores available for each question range from 4, meaning perfect function, to 0, meaning loss of function. These question scores are summed to make the ALSFRS-R score, which has a maximum of 48 and a minimum of 0. ALSFRS-R is often the primary outcome measure in clinical trials, but not without some controversy over whether this is appropriate \cite{vaneijkOldFriendWho2021}. Since the questions in the ALSFRS-R span a wide range of domains, there are arguments to limit use of the combined score and instead focus on domain-specific scores, such as the "limb movement score" or the "bulbar score" \cite{rooneyWhatDoesALSFRSR2017}.

MND patients can also be categorised into disease stages, which is supposed to reveal how far along a patient is into their disease course \cite{feldmanAmyotrophicLateralSclerosis2022}. The most popular staging systems are the King's Clinical Staging System \cite{rocheProposedStagingSystem2012} and the ALS Milano-Torino Staging System \cite{chioDevelopmentEvaluationClinical2015}, although they are not yet in widespread clinical use \cite{fangComparisonKingMiToS2017}.

The only treatment available for MND patients in the United Kingdom is Riluzole, an anti-glutamate agent which has been shown to increase median survival by 3 months \cite{millerRiluzoleAmyotrophicLateral2012, hinchcliffeRiluzoleRealworldEvidence2017}, although there is debate on whether this increase survival is only possible in the advanced stages of ALS which not all patients will reach \cite{andrewsRealworldEvidenceRiluzole2020}.

Outside of the United Kingdom, Edaravarone is a licensed drug that has shown promise in slowing disease progression. However, the trial has been critised for its restrictive inclusion criteria and there have been concerns raised over its safety \cite{witzelSafetyEffectivenessLongterm2022}.



\textbf{therapies}

\textbf{trials}

\begin{itemize}
    \item General description of MND: fatal neurodegenerative condition
    \item Survival time
    \item Incidence
    \item Types of MND: ALS, PLS, PMA, etc
    \item Causes
    \item Symptoms: motor, cognitive and behavioural
    \item Diagnostic delay
    \item Path to diagnosis
    \item Treatments
    \item Prognosis options
\end{itemize}

\noindent Imaging in MND:
\begin{itemize}
    \item Only used in the clinical pathway for diagnosis: Ruling out ALS mimics, El Escorial scale, imaging to rule out other conditions
    \item Neuroimaging-derived measures have been shown to be associated with patient prognosis - insert citations here
    \item Machine learning for images, and specifically medical imaging, has been shown to be useful for prognosis in other diseases, and to some extent for MND
    \item Imaging can be difficult in MND later in the disease due to muscle atrophy and difficulty in positioning as well as swallowing and saliva problems
    \item A growing area of machine learning is multimodal data fusion, where different types of data are combined to improve the performance of the model.
    \item Imaging has not been included yet into the clinical pathway for prognosis, but perhaps it has a role to play when combined with clinical data through multimodal data fusion AI
\end{itemize}


This work aims to create a prognostic tool for MND using multimodal data fusion AI, combining clinical, imaging and other data types together at the diagnostic appointment


\section{Motivation}

\begin{itemize}
    \item Why MND prognosis? Patients want it, good for clinical trial stratification, good for planning treatment
    \item Why imaging? Already being taken at diagnosis for differential diagnosis, might as well use it for prognosis - healthcare economics, no extra data collection. Also has shown promise in the literature. A lot fo machine learnign literature on imaging too - good for deep learning because it's so big and complex.
    \item Why data fusion? MND is multifactorial, so it makes sense to use all the data available to us. Also, it's a growing area of machine learning, and has shown promise in other diseases.
\end{itemize}

\section{Project Aims}

\begin{enumerate}
    \item Aim 1: investigate value of imaging in non-ML data fusion prognosis
    \item Aim 2: explore methods for multimodal data fusion in the literature and apply to larger multimodal neurodegenerative disease datasets
    \item Aim 3: apply to MND prognosis prediction with clinical and imaging data: find the value of imaging and the best way to use it (whole brain, regions of interest, texture analysis, etc
    \item Aim 4: Add more modalities such as fluid biomarkers and NLP from radiological reports
    \item Aim 5: Create the optimal multimodal data fusion model for MND prognosis and assess added values of different modalities
\end{enumerate}

\section{Upgrade Thesis Outline}