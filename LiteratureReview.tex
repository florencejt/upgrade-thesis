\chapter{Literature Review}
\label{literature_review}

This chapter contains a review of literature on prognosis of motor neuron disease, looking at both prognostic factors from various data types, and also attempts to predict prognosis using machine learning.

\section{Prognostic Factors}

How can prognosis be measured? Survival, rate of progression, future ALSFRS-R, future need for therapies, composite end point of NIV, tracheostomy, and death.

\subsection{Clinical}

A large meta-analysis in 2021 collated research studies on non-genetic prognostic factors in ALS associated with survival risk~\cite{suPredictorsSurvivalPatients2021}.
The review calculated and reported the hazard ratios (HRs) for each factor which had at least 3 studies reporting on it. Heterogeniety was assessed using the I-squared statistic, and sensitivity analysis was performed to assess the robustness of the results.
None of the sensitivity analyses showed a significant change in the results, and the authors concluded that the results were robust.

Starting with demographic factors, the meta-analysis found a higher age of symptom onset is associated with a higher risk of death (HR = 1.03)~\cite{suPredictorsSurvivalPatients2021}. However, age of onset is an imprecise measure because it is diffiuclt for patients to pinpoint their exact date of symptom onset, since symptoms often arise gradually.
It is common in clinical records for the onset date to be the first date of the month or even the first day of the year if the patient is having trouble remembering. Nevertheless, the harmful effect of older age at onset is consistent with much of literature (FIND CITATIONS HERE).
The meta-analysis also found that single, as opposed to married, patients have a worse prognosis (HR = 1.73), and that being a current smoker (HR=1.37) or a former smoker (HR=1.16) are both negative prognostic factors~\cite{suPredictorsSurvivalPatients2021}.

Rapid weight loss is a common feature of MND because of difficulties eating and swallowing, decreased appetite, and loss of muscle mass. A higher body-mass index (BMI) at diagnosis is associated with a lower risk of death (HR = 0.97) from 17 different studies in the meta-analysis ~\cite{suPredictorsSurvivalPatients2021}.
Another smaller meta-analysis specifically on BMI also found that higher BMI at baseline is protective HR=0.96)~\cite{dardiotisBodyMassIndex2018}, also backed up by a large population study of 1,809 patients from China (BMI >= 25 kg/m2, HR=0.36)~\cite{gaoEpidemiologyFactorsPredicting2021}.
However, there are some studies that have found no association between baseline BMI and survival, rather that the most important BMI prognostic factor is the rate of change in BMI~\cite{jawaidDecreaseBodyMass2010}.
Specifically, a decrease in BMI both 5 years and 10 years before symptom onset has been associated with shorter survival, in a study of N=381 ALS patients~\cite{goutmanBodyMassIndex2023}.
Since BMI is a simplistic measure that fails to capture body composition~\cite{rothmanBMIrelatedErrorsMeasurement2008}, Lindauer and colleagues used MRI of the knees and diaphragm to measure levels of subcutaneous and visceral fat in 62 ALS patients.
They found that higher subcutaneous fat is associated with higher ALSFRS-R and lower rate of ALSFRS-R decrease~\cite{lindauerAdiposeTissueDistribution2013}. Visceral fat was not associated with either measures.

The appearance of cognitive and behavioural impairment can also have an impact on survival.
Executive dysfunction, the inability to manage cognitive tasks, is associated with fast disease progression~\cite{elaminExecutiveDysfunctionNegative2011} and shorter survival in the meta-analysis (HR=2.1)~\cite{suPredictorsSurvivalPatients2021}.
Moreover, the meta-analysis found that the presence of FTD (HR=2.98) and non-specific dementia (HR=1.41) are both associated with shorter survival~\cite{suPredictorsSurvivalPatients2021}.

The El Escorial criteria are used to diagnose ALS, and a patient being assigned to the "probable" category as opposed to the "definite" category is associated with longer survival (HR=0.73), and "possible" is associated with even longer survival (HR=0.60)~\cite{suPredictorsSurvivalPatients2021}.
Definite ALS patients progressing faster also found in a large multi-centre study~\cite{westenengPrognosisPatientsAmyotrophic2018}. However, in the large Chinese cohort of 1,809 patients, there was no significant relationshop between El Escorial and survival~\cite{gaoEpidemiologyFactorsPredicting2021}.

The meta-analysis also considered the association of survival with site of onset, where motor symptoms first appear.
In comparison with spinal onset, which is the most typical ALS onset site, both respiratory onset (HR=2.2) and bulbar onset (HR=1.35), meaning the first symptoms are in the mouth and throat, are associated with shorter survival~\cite{suPredictorsSurvivalPatients2021}.
The onset sites that are protective compared to spinal onset are flail arm or leg onset (HR=0.61) and predominantly upper or lower motor neuron (pUMN or pLMN) (HR=0.32).
The speed of progression of the disease is also a prognostic factor. Fujimura-Kiyono and colleagues found that, in a cohort of 150 sporadic ALS patients, a shorter interval between the first motor onset and the next site involvement is associated with shorter survival, independent of what those onset sites are~\cite{fujimura-kiyonoOnsetSpreadingPatterns2011a}.

A higher ALSFRS-R at baseline is associated with longer survival (HR=0.96), and a higher rate of decline in ALSFRS-R from onset to diagnosis is associated with shorter survival (HR=1.48 for categorical, HR=2.37 continuous)~\cite{suPredictorsSurvivalPatients2021}.
The rate of decline in ALSFRS-R from onset to diagnosis is also called the progression rate to baseline, or PRB, and is calculated as
\begin{equation}
    PRB = \frac{48-\textnormal{ALSFRS-R}(t_{diag})}{t_{diag}-t_{onset}},
\end{equation}\label{eq:PRB}
where $t_{diag}$ and $t_{onset}$ are the dates of MND diagnosis and symptom onset respectively, and 48 is the maximum score of ALSFRS-R.

Other clinical factors associated with shorter survival are lower forced vital capacity (FVC), and a shorter time between symptom onset and diagnosis~\cite{suPredictorsSurvivalPatients2021}.
When the diagnostic delay is longer than one year, Su and colleagues found it to be associated with longer survival in their meta-analysis (HR=0.39)~\cite{suPredictorsSurvivalPatients2021}, and results from the large Chinese cohort agree that an onset shorter than one year is harmful to survival risk (HR=3.43).
This is probably because the longer the delay, the more likely it is that the patient has a slowly progressing form of MND that is not obviously ALS in diagnostic tests.

Statins, a drug used to inhibit cholesterol synthesis, has also been investigated as a prognostic factor in MND.
Su and colleagues found that taking or not taking statins has no significant effect on survival, based on 3 papers that all reported non-significant HRs~\cite{suPredictorsSurvivalPatients2021}.
However, a study published after the meta-analysis looked not only at statin use but also at the dose of statins and the duration of statin use.
Weisskopf and colleagues found that, in a cohort of 948 ALS patients, taking statins for under 3 years is protective for survival (HR=0.77), but that there is no significant protection or harm when the duration of statin use is over 3 years~\cite{weisskopfStatinMedicationsAmyotrophic2022}.
Additionally, they found that taking low-potency statins compared to not taking any statins is protective for survival (HR=0.82), but they found no signficant effects with higher-dose statins.
They concluded that the statins may have a protective effect on ALS survival, if the underlying reasons for taking statins do not necessitate high dose and long duration of use, which would imply a more severe cardiovascular condition that may be harmful to survival.

Finally, the meta-analysis found that taking Riluzole is associated with longer survival (HR=0.80)~\cite{suPredictorsSurvivalPatients2021}.

\subsection{Genetic}

Genetic testing after diagnosis is becoming more common now that the extent of the genetic contribution to MND is better understood and that therapeutics are being developed to target specific genetic mutations~\cite{efnstaskforceondiagnosisandmanagementofamyotrophiclateralsclerosis:EFNSGuidelinesClinical2012}.
Increased data availability of genetic markers has led to a number of studies on the prognostic value of genetic markers in MND.
A network meta-analysis on genetic factors associated with survival in ALS found that the C9orf72 repeat expansion is associated with shorter survival compared to no known variants (HR=1.6)~\cite{suGeneticFactorsSurvival2022}.
This is consistent with another study of cohort of 1,245 ALS patients that found that, in a multivariable Cox regression model, the C9orf72 repeat expansion is associated with shorter survival (HR=1.65)~\cite{chioAssociationCopresencePathogenic2023}.
Also associated with shorter survival is ATXN2, a CAG repeat expansion usually associated with spinal onset ALS (HR=3.6), and a mutated FUS (fused in sarcoma) (HR=1.8)~\cite{suGeneticFactorsSurvival2022}.

\subsection{Fluids}

Fluid biomarkers are measurements of proteins, metabolites, or other molecules in the blood/serum, cerebrospinal fluid (CSF), or urine that can be used to diagnose and monitor disease.

From the meta-analysis of non-genetic progostic factors, higher levels of creatine kinase and creatinine in serum indicate longer survival, with HR=0.68 and HR=0.64 respectively~\cite{suPredictorsSurvivalPatients2021}.
Harmful fluid factors in this meta-analysis  include higher levels of neurofilament light chain (NfL) in CSF (HR=6.8), NfL in serum (HR=3.7) and albumin in serum (HR=1.52).

The most studied fluid biomarker in MND is NfL, which is a protein that is released into the CSF and blood during neuroaxonal injury, which occurs in neurodegeneration.
NfL is best measured in CSF, and the CSF is extracted by performing a lumbar puncture on the patient, which is an invasive procedure and difficult to perform when the patient is further progressed in the disease~\cite{sturmeyBloodBiomarkersALS2022}.
However, NfL can also be measured in blood, although it is harder to detect because the concentration is lower than in CSF. This is also reflected in the hazard ratios for NfL in the meta-analysis, where the HR for NfL in CSF is higher than in serum.

NfL levels are elevated in ALS patients compared to healthy controls, but they are also elevated in other neurodegenerative diseases~\cite{huangLongitudinalBiomarkersAmyotrophic2020}.
The levels of NfL rise presymptomatically in ALS~\cite{benatarValidationSerumNeurofilaments2020}, and studies have shown that NfL concentrations plateau around a year after symptom onset~\cite{benatarNeurofilamentsPresymptomaticALS2019, benatarValidationSerumNeurofilaments2020, thompsonMulticentreAppraisalAmyotrophic2022}.

In a review by Irwin and colleagues, they showed that the general consensus in the literature among over 20 studies is that higher baseline NfL concentration is associated with shorter survival~\cite{irwinFluidBiomarkersAmyotrophic2024}, although there is less consistency in results when correlating NfL with clinical measures such as ALSFRS-R.

Dreger and colleagues investigated the association between NfL and D50-derived disease aggressiveness and disease accumulation in a cohort of 156 ALS patients~\cite{dregerCerebrospinalFluidNeurofilament2021}.
They found that higher baseline NfL was significantly associated with higher disease aggressiveness, independent of gender, site of onset, and disease accumulation. The independent association of NfL with disease accumulation corroborates the idea that NfL plateaus quickly after onset, and so has limited staging value.
These significant associations were influenced by age, presence of dementia, and the laboratory where the NfL was measured, which highlights the importance of considering these factors when interpreting NfL results.


\subsection{Neuroimaging}

Structural MRI of the brain is often conducted as part of the differential diagnosis of MND, in order to rule out mimic diseases.
Imaging is not used for prognosis and progression monitoring clinically because, as the disease progresses, it becomes difficult and uncomfortable for the patient to lie still in the MRI scanner, and the MRI signal is sensitive to motion artefacts.
Therefore, the majority of imaging studies in MND are cross-sectional, and the number of longitudinal studies is limited.
However, there have been a number of studies that have found associations between baseline imaging measures and prognosis in MND.
As a disclaimer, the vast majority of these studies are only on ALS patients, rather than the full spectrum of MND, and the results may not be generalisable to the whole MND population.

A critical review of imaging studies in ALS up to 2014 found that some of the most consistent limitations affecting these studies were small sample sizes and poor patient characterisation, meaning that information on clinical phenotypes and genetic status was often missing, which could have affected the significance of results~\cite{bedeLessonsALSImaging2014}.
Especially noted was the inconsistency that is common in ALS imaging studies, where the same imaging measure is found to be both increased and decreased in different studies.
However, some consistent findings have emerged, such as the finding that the corticospinal tract (CST) is affected in ALS, and that the motor cortex is affected in ALS, which is consistent with the clinical presentation of the disease.

Prognosis is studied in ALS by looking at the relationship between imaging measures and clinical measures of progression.
These measures of progression include ALSFRS-R, the rate of decline in ALSFRS-R, and survival time (comparing short and long survivors, for example).
Although, correlating imaging measures with ALSFRS-R and extracting prognostic meaning is not an ideal methodology because the ALSFRS-R is a measure dominated by the effects of lower motor neuron degeneration, which is not well captured in brain MRI~\cite{bedeLessonsALSImaging2014}.

In this section, we are explore brain regions implicated in MND prognosis, grouped by their location in the brain.
However, some studies have focused on overall brain structure differences by totalling the volume of all grey matter (GM) or white matter (WM) in the brain, and found that lower total baseline GM volume is associated with faster progression (N=29)~\cite{elmendiliAssociationBrainUpper2023}.
When the GM was parcellated further, the only discriminating factor in ROC analysis between fast and slow progressors was the cortical GM volume.
This is consistent with a study of 32 ALS patients who underwent multimodal MRI at three time points, where the authors concluded that changes to WM generally occur before diagnosis, making them a potential biomarker for early diagnosis, and that GM changes occur after diagnosis, making them a potential biomarker for prognosis~\cite{bedeLongitudinalStructuralChanges2018}.
Other studies have used diffusion tensor imaging (DTI) to measure the microstructure of the brain, and looked at the the resulting metrics averaged over the whole brain.
The main DTI measures are fractional anisotropy (FA), mean diffusivity (MD), axial diffusivity (AD), and radial diffusivity (RD).
Lower overall FA has been associated with faster progression in a study of 67 patients~\cite{sendaStructuralMRICorrelates2017} and 6 patients~\cite{baldaranovLongitudinalDiffusionTensor2017}.
However, Trojsi and colleagues found no differences in GM or WM damage between fast and slow progressors, both measured by structural MRI and by DTI metrics (N=54)~\cite{trojsiRestingStateFunctional2021}.
In their study, the only difference between fast and slow progressors was in functional connectivity, where fast progressors had decreased functional connectivity in the motor and extra-motor networks, and increase functional connectivity in the salience network.
This inconsistency of findings is common in ALS imaging studies, and is likely due to the small sample sizes and the inconsistency of progression measures used in the studies.

\subsubsection*{Motor Cortex and Corticospinal Tract}
MND is characterised by upper and lower motor neuron degeneration, so changes in the motor cortex and the corticospinal tract (CST) are expected to be affected by the disease.
Particularly in the motor cortex, the motor band sign has been implicated in affecting prognosis, and the motor band sign is a signal attenuation or hypointensity in the shape of a ribbon at the posterior border of the precentral gyrus, which is seen on T2-weighted MRI and also on susceptibility-weighted imaging (SWI) and quantitative susceptibility mapping (QSM)~\cite{bollHypointensityMotorCortex2019}.
Motor band sign intensity was found to be a predictor of shorter survival in a study of 73 ALS patients (HR=2.97)~\cite{rizzoDiagnosticPrognosticValue2020}, and the difference in motor band sign intensity after 18-month followup was significantly correlated with ALSFRS-R disease progression, but only in a study of 7 patients~\cite{bollHypointensityMotorCortex2019}.

One of the most studied area of the brain in ALS is the CST, which is the main motor pathway in the brain, connecting the motor cortex to the spinal cord.
Most studies have found that higher FA in the CST is a protective factor in prognosis.
Associating these DTI metrics with ALSFRS-R, higher FA in the CST has been associated with higher baseline ALSFRS-R, both bilaterally in a large study (N=253)~\cite{mullerLargescaleMulticentreCerebral2016} and only on the left side in a study of 33 patients~\cite{liBrainstemInvolvementAmyotrophic2021}.
Lower average CST FA has been significantly correlated with higher progression rate in a study of 24 patients~\cite{agostaMRIPredictorsLongterm2010}, and lower FA and higher RD in both the left and right CSTs has been associated with PRB in a study of 60 patients~\cite{menkeWidespreadGreyMatter2014}.
A faster rate of FA decline in CST has also been associated with faster ALSFRS-R decline in the first 8 months post-diagnosis (N=66)~\cite{kalraProspectiveHarmonizedMulticenter2020}.
Finally, higher FA in the left CST has been associated with longer survival (HR=0.97) in a study of 24 patients~\cite{agostaMRIPredictorsLongterm2010}.

White matter tracts other than the CST have also been found to be associated with prognosis in MND, and Burgh and colleagues described widespread loss of white matter integrity being characteristic of ALS (N=292)~\cite{burghMultimodalLongitudinalStudy2020}.
Similar to the CST, high FA has been found to be a protective factor in prognosis in the posterior limb of the internal capsule (as well as low MD, N=41)~\cite{grolezMRICervicalSpinal2018} and in the right superior longitudinal fascicle (as well as low RD, N=60)~\cite{menkeWidespreadGreyMatter2014}.
When disease aggressivness has been estimated by the D50 model, studies have found that higher disease aggressiveness is associated with cerebral white matter density decreases in tracts connecting the frontal, parietal, and occipital lobes (N=85)~\cite{steinbachApplyingD50Disease2020}, and with MD and AD elevations in the fronto-parietal tract (N=145)~\cite{steinbachDiseaseAggressivenessSignatures2021}.

\subsubsection*{Cortical Thickness}
Cortical thickness (CT) is another measure of brain structure that has been associated with prognosis in MND, and is the distance between the pial surface and the grey-white matter boundary, and is often measured using T1-weighted MRI.
CT loss has been associated with faster progression in MND, in the temporal lobe (N=20, N=45)~\cite{dambrosioFrontotemporalCorticalThinning2014, verstraeteStructuralMRIReveals2012} and in the frontal lobe (N=45)~\cite{verstraeteStructuralMRIReveals2012}.
However, when considering a measure of disease aggressiveness estimated from the D50 model, Dieckmann and colleagues found no association with CT volumes at baseline in a cohort of 100 ALS patients~\cite{dieckmannCorticalSubcorticalGrey2022}.
A caveat to that finding is that they did not include any patients with cognitive deficits, so the results may not be generalisable to the whole ALS population since cognitive deficits are common in ALS.
An interesting finding from Burgh and colleagues (N=292) found that longer survivors had widespread cortical thinning at diagnosis, but this stayed constant over time, whereas short survivors had less CT thinning at diagnosis, but then more extensive changes to CT over time (N=292)~\cite{burghMultimodalLongitudinalStudy2020}.

\subsubsection*{Subcortical Structures}
Subcortical structures in the brain that have been associated with ALS prognosis are the thalamus, the basal ganglia, and the amygdala.
Atrophy in the thalamus has been associated with faster decline of ALSFRS-R (N=67)~\cite{sendaStructuralMRICorrelates2017}, and right thalamus volume was the only subcortical brain region with significant association with disease aggressivness from D50 model (N=100)~\cite{dieckmannCorticalSubcorticalGrey2022}.
Additionally, texture analysis metrics from SWI in the left thalamus were significantly correlated with progression rate (N=17)~\cite{johnsQuantifyingChangesSusceptibility2019}.

In the basal ganglia, GM atrophy in the left caudate and right putamen (N=17)~\cite{agostaLongitudinalAssessmentGrey2009} and the caudate nucleus (N=67)~\cite{sendaStructuralMRICorrelates2017} have been associated with faster progression.
Shorter survival in a cohort of 112 ALS patients has been associated with an overall smaller basal ganglia and a smaller amygdala at baseline in a Cox model~\cite{westenengSubcorticalStructuresAmyotrophic2015}. However, when adjusted for age of onset, these significances were lost.
Finally, in a cohort of 157 patients, Ishaque and colleagues found significant texture changes, derived from texture analysis on DTI, in the basal ganglia and hippocampus of short surviving patients, but not in longer surviving patients~\cite{ishaqueEvaluatingCerebralCorrelates2018}.
The longer surviving patients had texture changes restricted to motor regions.

\subsubsection*{Hippocampus}
Further studies have looked at the hippocampus with mixed results.
A smaller hippocampus was associated with shorter survival in the Cox model of 112 patients~\cite{westenengSubcorticalStructuresAmyotrophic2015}, but this significance was lost when adjusted for age of onset, as with the basal ganglia and amygdala.
Abdulla and colleagues found no correlation between hippocampal volume and ALSFRS-R or rate of decline in ALSFRS-R (N=58)~\cite{abdullaHippocampalDegenerationPatients2014}, and Muller and colleagues found no correlation between hippocampal FA and ALSFRS-R (N=253)~\cite{mullerLargescaleMulticentreCerebral2016}.
Dieckmann and colleagues found that decreased bilateral hippocampal volume was associated with disease accumulation, but not disease aggressiveness, in a cohort of 100 ALS patients~\cite{dieckmannCorticalSubcorticalGrey2022}.
Progression rates significantly negatively correlated with local shape distances in the right hippocampus in a cohort of 32 ALS patients, meaning that the more the shape of the right hippocampus deviates from the average shape, the faster the progression rate~\cite{taeShapeAnalysisSubcortical2020}.
Finally, Stoppel and colleagues found that increased hippocampal activation in resting-state fMRI was associated with lower ALSFRS-R, meaning a more advanced disease (N=14)~\cite{stoppelStructuralFunctionalHallmarks2014}.

\subsubsection*{Frontal Lobe}
Since ALS is closely related to FTD, the frontal lobe and fronto-temporal lobe have been studied in relation to prognosis in MND.
GM atrophy in the frontotemporal lobe has been associated with medium and fast progression in a study of 67 patients, and this was not seen in slow-progressing patients~\cite{sendaStructuralMRICorrelates2017}.
Additionally, decreased FA in the frontotemporal lobe has been associated with rapid progression in the same study~\cite{sendaStructuralMRICorrelates2017}.
This result was also found in the frontol lobe, where a faster decline in FA was associated with faster decline in ALSFRS-R in a study of 66 patients~\cite{kalraProspectiveHarmonizedMulticenter2020}.
However, Muller and colleagues found no correlation between frontol areas FA and ALSFRS-R (N=253)~\cite{mullerLargescaleMulticentreCerebral2016}.
Finally, fast progressors have decreased functional connectivity in middle frontal gyri and paracentral lobule (N=45)~\cite{trojsiRestingStateFunctional2021}, and ALSFRS-R correlated with reduced connectivity in left sensorimotor cortex (N=26)~\cite{agostaSensorimotorFunctionalConnectivity2011}.

\subsubsection*{Ventricles}
Ventricles are enlarged when the brain atrophies, and larger ventricular volume has been associated with lower baseline ALSFRS-R in a study of 112 patients~\cite{westenengSubcorticalStructuresAmyotrophic2015}.

\subsubsection*{Brain Stem}
The brain stem is another area of the brain with mixed results in relation to prognosis in MND.
Since the brain stem is involed in the regulation of breathing, and the most common cause of death in MND is respiratory failure, it is expected that the brain stem would be a candidate prognostic marker in MND.
Baseline medulla oblongata volume significantly predicted short versus long survival in a cohort of 60 ALS patients~\cite{milellaMedullaOblongataVolume2022}.
However, Steinbach and colleagues found that brain stem GM density has no affect on aggressivess disease from D50~\cite{steinbachApplyingD50Disease2020}, and no correlation between brain stem FA and ALSFRS-R was found in a study of 253 patients~\cite{mullerLargescaleMulticentreCerebral2016}.

\subsubsection*{Other measures}
Finally, other imaging measures have been associated with prognosis in MND.
For one, brain age is a measure of how much the brain has aged compared to the average brain of the same chronological age, and is calculated using machine learning models trained on healthy brain MRI data.
An increased brain age, meaning the brain looks older than it should, has been associated with faster progression in a study of 112 patients~\cite{hermannCognitiveBehaviouralNot2022}.
Interestingly, significant brain age changes were only found in ALS patients with cognitive and behavioural impairment, and not in those without. This could signify that obvious brain changes are only seen in patients with cognitive and behavioural impairment, and that the brain changes in patients without cognitive and behavioural impairment are more subtle.
Many of the imaging studies discussed in this section either did not include patients with cognitive and behavioural impairment, or did not report on the presence of cognitive and behavioural impairment, so this could be a confounding factor in the results of these studies.

Magnetic resonance spectroscopy (MRS) is a technique that can be used to measure the concentration of metabolites in the brain, and has been used to study prognosis in MND.
The N-acetylaspartate (NAA) to choline (Cho) ratio is a measure of neuronal health, and lower NAA/Cho in the primary motor cortex has been associated with shorter survival in a Cox hazard survival analysis of 63 patients~\cite{kalraCerebralDegenerationPredicts2006}.
Even when ALSFRS-R and FVC were included in the model, NAA/Cho was the significant predictor of survival (HR=0.14).

In summary, advanced imaging studies provide valuable insights into the diverse structural and functional associations of MND prognosis.
While findings highlight the involvement of various brain regions and white matter tracts, including the corticospinal tract and subcortical structures, challenges such as inconsistent results and confounding factors show the importance of analysing imaging data with better patient characterisation, through clinical data.

\subsection{Spinal Cord Imaging}

Imaging of the spinal cord can also be conducted during the differential diagnosis to rule out alternative pathologies~\cite{elmendiliSpinalCordImaging2019}.
Due to its small axial size and the proximity of the lungs and heart, the spinal cord is difficult to image and is affected by many movement artefacts.
Although it is usually qualitatively interpreted, the cross section area (CSA) of the spinal cord has been quantitatively measured in a number of studies, and has been associated with prognosis in MND.

In a study of 43 ALS patients, Branco and colleagues found significant corelations between baseline ALSFRS-R and cervical spine CSA, and also between disease duration and CSA~\cite{brancoSpinalCordAtrophy2014}.
Moreover, Grolez and colleagues found that a smaller reduction in cervical spine volume over 3 months is associated with longer survival in a study of 41 patients~\cite{grolezMRICervicalSpinal2018}.
However, in a study of 218 MND patients, including ALS, PLS, and PMA, the CSA of the cervical spine only correlated with the baseline ALSFRS-R of PLS and PMA patients, although there was no longitudinal spinal atrophy for the PLS patients~\cite{vanderburghCrosssectionalLongitudinalAssessment2019}.


\subsection{Prognostic Models}
There are limited clinical tools for prognosis prediction in MND. The most common way for progression to be assessed is through the progression rate in Equation~\ref{eq:PRB}, which can be generalised to the progression rate to any time point, not just diagnosis.
Extrapolating this progression rate to predict future ALSFRS-R scores is a common way to predict future progression, called the ``pre-slope model``, but it is not a perfect method since it assumes that ALSFRS-R decline is linear.

To overcome these issues, the D50 model was developed, which fits a sigmoidal curve to a patient's ALSFRS-R timepoints, yielding an individualised prediction of future ALSFRS-R scores
~\cite{poesenNeurofilamentMarkersALS2017, steinbachApplyingD50Disease2020}.
The D50 model can provide a measure of disease aggressiveness, which is the estimated rate of functional loss from the sigmoidal curve, and disease accummulation, which is the patient's position on the D50 curve independent of time.

A non-linear extension of the D50 model was proposed by Ramamoorthy and colleagues, where Gaussian processes are used to non-parametrically cluster patients into non-linear ALSFRS-R trajectories~\cite{ramamoorthyIdentifyingPatternsAmyotrophic2022}.
In their model development, they found that many of the patients in their cohort had non-linear ALSFRS-R trajectories (convex, concave, sigmoidal) and that the slope of ALSFRS-R in the first year of the disease is not sufficient to accurately predict future ALSFRS-R scores.

Finally, a popular model for predicting survival probability in ALS is the ENCALS model~\cite{westenengPrognosisPatientsAmyotrophic2018}.
Using data from 14 European ALS centres and over 11,000 patient records, Westeneng and colleagues developed a multivariable Royston-Parmar model to predict a survival probability function for an individual ALS patient from baseline information.
They selected the predictors through backward elimination and bootstapping, and predictors present in more than 70\% of the bootstrapped models were included in the final model.
The harmful predictors included in the final model are bulbar onset (HR=1.71), age of onset (HR=1.03), El Escorial definite ALS (HR=1.47), higher PRB (HR=6.33), presence of FTD (HR=1.34), and C9orf72 repeat expansion (HR=1.45).
The protective predictors are a longer diagnostic delay (HR=0.52) and a higher FVC (HR=0.99).

The ENCALS model is available by request of medical doctors, and was validated using a leave-one-site-out protocol.
The authors demonstrated the model's ability to predict survival by using data from Stephen Hawking, a famous physicist who lived with ALS for a startling 55 years after diagnosis, and found that the model predicted his survival probability well.
The predicted probability of surviving over 10 years was 94\% and the interquartile range of predicted year of death was 1981 to 2011, which is consistent with the year of Stephen Hawking's tracheostomy in 1985 (which is the endpoint of the model's prediction: NIV for more than 23 hours a day, tracheostomy, or death).


\section{Machine Learning for Prognosis}


Link to previous section: we know a lot of prognostic factors in MND, could tools be developed for clinicans to use these factors to automate or improve clinical decisions like diagnosis and prognosis?


\begin{itemize}
    \item What is machine learning?
    \item How is it different to statistics?
    \item What are the different types of machine learning?
    \begin{itemize}
        \item Supervised learning
        \item Unsupervised learning
    \end{itemize}
    \item In MND it can be done for prognosis: which is what we're looking at but also for diagnosis
    \item Honourable mention of some studies that have used machine learning for diagnosis
\end{itemize}



\subsection{Clinical}

% point of section: lots of ML has been done on MND with very varied designs and outcomes, with pros and cons to each
% Call to action: there is less focus on predicting actual survival

As mentioned earlier, prognosis can be defined as the prediction of the future course of a disease.
In context of MND, prognosis could be future ALSFRS-R scores, progression rate, survival time, or predicting time until a treatment is needed.

Predicting future progression rate, as calculated by the slope between the ALSFRS-R scores, is a popular prediction task in ML prognosis.
In 2011, a challenge was set called the DREAM Phil Bowen Prize4Life ALS Prediction Challenge, which asked entrants to use 3 months of clinical trials data to predict the progression rate in the next 9 months, measured as the slope between the ALSFRS-R scores at month 3 and at month 12~\cite{kuffnerCrowdsourcedAnalysisClinical2015}.
The dataset accompanying this challenge was the PRO-ACT database, which is the largest publicly available dataset of clinical trials data in MND with over 8,500 patients from multiple trials~\cite{atassiPROACTDatabaseDesign2014}.
This dataset has been widely used in ML studies for MND prognosis because of its size and longitudinal nature.
However, the data is taken from clinical trials, where and he patients were subject to strict inclusion and exclusion criteria, leading to younger patients wih fewer functional impairments.
This limits the generalisability of the results to the whole MND population.

% point of paragraph: challenge kicked off ML for prognosis properly, found that tree-based models performed the best, deep learning still not useful. however, unideal study
The challenge had 37 entries and the results were summarised by Küffner and colleagues~\cite{kuffnerCrowdsourcedAnalysisClinical2015}, who concluded that random forest and tree-based decision models performed the best in predicting future progression rate~\cite{hothornRandomForest4LifeRandomForest2014}.
By comparing models and results from the entries, some interesting findings emerged, such as high variability in individual ALSFRSr scores being a strong predictor of progression rate~\cite{hothornRandomForest4LifeRandomForest2014} and previously unidentified progression biomarkers such as blood pressures and uric acid.
Ensemble models, which are a combination of predictions from multiple models, have shown promise in the PRO-ACT data as well in a post-challenge study of 17 classical ML models ~\cite{turabiehMachineLearningEmpowered2024}.
Since the challenge took place before deep learning became as popular as it is now, Pancotti and colleagues revisited the challenge in 2022 using deep learning models~\cite{pancottiDeepLearningMethods2022}, finding that deep learning performed similarly to classical ML models.
It is possible that more data or a different task is needed to show the benefit of deep learning in MND prognosis.

% weird to predict a linear slope using ML, especially a slope over 9 months which smooths out a lot of the individual patient trajectories
While the Prize4Life challenge was a great catalyst for ML research in MND, there are limitations in predicting a rate of functional decline over 9 months.
Condensing 9 months of progression into a single slope may oversimplify the disease course, and training ML models to predict a linear slope may not be the best approach, especially since Ramamoorthy and colleagues showed how non-linear ALSFRS-R trajectories are very common~\cite{ramamoorthyIdentifyingPatternsAmyotrophic2022}.

% taylor et al predicting just future alsfrs-r, not the slope, and comparing to pre-slope model which assumes linear decline

Instead using PRO-ACT to predict progression rate, Taylor and colleagues predicted future ALSFRS-R scores using random forests~\cite{taylorPredictingDiseaseProgression2016}.
They found that random forest performed better than the pre-slope model, especially when predicting later into the disease course.
This shows the usefulness of a non-linear model in predicting functional decline in MND .

% point of paragraph: what about classifying by fast or slow progression?
Predicting fast versus slow progression is another common prognosis task, where labels ``fast`` or ``slow`` are assigned to patients by imposing some threshold on the progression rate~\cite{ongPredictingFunctionalDecline2017, dinabduljabbarPredictingAmyotrophicLateral2023}.
Performance is usually better in this task than predicting the actual progression rate, because the task is less granular and the classes are more balanced.
Additionally, stratifying patients by progression rate can be used as part of the model's pipeline.
Pires and colleagues found that they could improve the accuracy of predicting need for NIV by stratifying patients into low, medium, and high progression rate, and then training separate models for each strata~\cite{piresPredictingNoninvasiveVentilation2018}.
Similarly, Halbersberg and colleagues clustered patients by ``deterioration patterns`` based on their early ALSFRS-R scores, and trained separate classifiers for each cluster~\cite{halbersbergTemporalModelingDeterioration2019}.
The success of these stratification approaches suggests that the heterogeneity of MND may be too great for a single model to predict progression rate accurately, and that stratifying patients into groups may be a better approach.

However, the threshold for fast and slow progression is arbitrary, and the threshold used in these studies may not be the same as the threshold used in a clinical setting.
A more data-driven way of grouping patients is to use clustering, which is a type of unsupervised learning where the model groups patients into clusters based on their features.
Grollemund and colleagues used UMAP (uniform manifold approximation and projection) to reduce the dimensionality of patient data, and then coarsely divided the lower-dimensional space (or latent space) into three tiers of 1-year survival risk~\cite{grollemundDevelopmentValidation1year2020}.
They found that this approach outperformed random forest and logistic regression in predicting 1-year survival, even though the latent space had no knowledge of the survival times of the patients in the training data.

Another goal in MND prognosis is to predict time to treatment, such as time to NIV or time to PEG (percutaneous endoscopic gastrostromy).
The IDPP Clef challenge is a recent challenge in 2022 that focused on predicting risks of clinical events and timings in MND, and included a dataset of 2559 patients from Italy and Portugal~\cite{guazzoOverviewIDPPCLEF2022}.
The challenge resulted in 4 papers, which all found that it was comparatively simple to predict the risk of clinical events, but not the timing of the events~\cite{brancoHierarchicalModellingALS2022, mannionPredictingRiskTime2022, trescatoBaselineMachineLearning2022, pancottiMultiEventSurvivalPrediction2022}.
Other attempts at predicting time to treatment have focused on predicting time to NIV, often classifying patients into those who will need NIV within 3 months, 3 to 6 months, and 6 to 12 months~\cite{carreiroPrognosticModelsBased2015, pereiraPredictingAssistedVentilation2019}.
The timing of medical interventions is a difficult task to predict because different clinics have different treatment strategies and different interpretations of the same clinical guidelines.
Predicting the assessment outcomes that lead to the decision to start treatment may be a more useful task.

% point of paragraph: machine learning approaches can be improved with explainability and setting logical rules, to make up for the small sample sizes
Models can also be made more clinically relevant by incorporating logical rules and explainability methods.
Tavazzi and colleagues used a dynamic bayesian network to simulate disease course according to the MiToS staging system~\cite{tavazziPredictingFunctionalImpairment2022}.
They applied clinical and biological logic to their graphical model by not letting impossible relationships between variables exist, such as the medical centre being dependent on the patient's sex.
This allows the model to learn from the data, but also to be guided by clinical and biological sense.
Additionally, the graph structure of the model can be used to explore dependencies between variables, and to identify which variables are most important in predicting the future disease course.

Explainable models are important in MND prognosis because of the small sample sizes and the need for clinical sense in the predictions.
Müller and colleagues used a deep learning longitudinal neural network to predict respiratory impairment in MND, and then used SHAP (SHapley Additive exPlanations) to explain the model's predictions~\cite{mullerExplainableModelsDisease2021}.
When looking at the most important features in the model through an explainability method called Shapley values, they found that better swallowing function indicated quicker respiratory decline, which is clinically unintuitive.
This brought the model's predictions into question, and shows the importance of explainability methods in MND prognosis.

Overall, predicting MND prognosis with ML and clinical data is a difficult task with many ways to approach it.
It is important to check that the model's predictions make sense in the context of the disease, and to consider the clinical relevance of the predictions.
In a survey of 242 Dutch ALS patients, when asked whether they would prefer to be told their predicted survival time or a survival category (slow, medium, fast), the majority of patients preferred the exact survival time~\cite{westenengPrognosisPatientsAmyotrophic2018} (supplementary materials).
However, none of the ML studies have attempted to predict exact survival time.
This could be a useful future direction for ML in MND prognosis, and could be a task more relevant to the patients than predicting progression rate.


\subsection{Imaging}

% point of section: less work has been done on imaging, and mostly done with derived features to help with small sample size issues
% imaging by itself has been shown to have limited usefulness

ML in MND with imaging is a diagnosis-heavy environment: shout-out to kushol 2023 who used vision transformer with structural MRI to predict ALS vs HC
- there is a lot of work in other diseases using state of the art machine learning with images
- processing images requires a lot of data because it needs more complex architectures to learn from the data

Only a few studies have used imaging in isolation to clinical data
- extracted features only

Li et al predicting fast vs slow baseline progression rate from connectivity, which is not clinically useful but backs up non-ML studies.
The SVM had balanced accuracy of 85\% that top contributing connection was left hippocampus to right thalamus, corroborating the involvement of subcortical regions in MND

Behler et al 2022 used bayesian classification with FA values in tracts associated in the pTDP-43 stages to assign patients to stages.
Again, limited clinical application, shown by there being no association between the DTI pTDP-43 stage and King's or MiToS staging systems

Thome 2022 used a random forest to classify between ALS and HC with structural and rs-fMRI, and then looked at associations with the classification-important features with progression rate and symptom duration.
Found no associations.

The most promising imaging study was querin 2017 who used FA of spinal cord and spinal cord atrophy in a cox survival model.
Found that MRI parameters were more predictive than clinical features, albeit in a small cohort of 49 patients

Conclusions are that imaging by itself has not really been used to predict prognosis - more investigated for diagnosis and understanding the disease.
And even then, it has shown limited usefulness, which is surprising given the wealth of prognostic factors found in neuroimaging studies in the previous section.
Perhaps the small sample sizes in these studies are limiting the usefulness of the imaging data, and that the imaging data needs to be combined with clinical data to be useful.

\subsection{Multimodal}

Maybe the imaging papers didn't work so well because they left out prognostic factors that have been shown to be important in the literature.
What if we combined the imaging data with the clinical data?

This can be done through a process called multimodal data fusion, where the data from different modalities is combined into a single model.
% Background on data fusion from JOSS paper?

Why would we want to do that for motor neuron disease
- multifactorial
    - already seen that in the prognostic factors listed above
    - no huge consensus on imaging factors so doing a data-driven approach with guidance from clinical data could help
- using machine learning could help understand underlying multifactorial mechanisms of MND progression

There have been a few studies that have used multimodal data for MND prognosis.

All of the studies have used features extracted from imaging, rather than the images themselves.

Papers that have concatenated clinical and imaging measures into their models:

Schuster 2017 classified patients as surviving or not surviving 18 months after diagnosis
- features chosen by comparing to healthy controls - is this the best way?
- training set performed best with clinical data and MRI, but testing set performed the same
    - small test size of n=12
- imaging and clinical data concatenated into a logistic ridge regression
- since the sample size was small, including more features may have led the model to overfit the training data, hence the lack of improvement in the testing set
- however, the clinical + mri model not being markedly worse is a good sign for the usefulness of imaging data, because it's possible that with a larger sample size, the imaging data would have been more useful

Kuan 2023
- n=172
- predicting probability of survival at multiple time points
- comparing clinical only, cortical thickness only, and both: concatenated before ML survival models
- best model was clinical only, but not statistically significantly better

Concatenation is a common pitfall when including data from two sources: curse of dimensionality
- there are data fusion methods to combat this
- what if we lowered the dimensionality of the data before using it in the model?

Dimensionality reduction:
Behler 2022
- PCA on concatenated features of cognitive, oculomotor, and DTI to cluster patients into neuropathological disease stages
- although this has limited clinical benefit, since neuropathological disease stage isn't useful in clinic and DTI is rarely obtained
    - using clustering allows for seeing interactions between the different modalities
    - which is important when doing multimodal data fusion because the different modalities may interact in complex ways that can't be captured by simple concatenation

Kmetzsch 2022
- a deep learning unsupervised method called a variational autoencoder
- mainly focused on disease progression of FTD
- using miRNA and T1w extracted volumes
- Variational autoencoder to lower the dimensions of the joint data
- Found that different modalities were more important at different stages of the disease
    - assessing added values of modalities unveils more about the disease

Supervised deep learning:
For example, van der Burgh 2017
- classifying 135 sporadic ALS patients into short, medium, and long survival
- using clinical characteristics, T1W extracted features, DWI extracted features
- fusion was done by training 3 unimodal neural networks separately, then concatenating their outputs as the input into a final neural network, which was trained on the output classifications of the training set
- multimodal model model performed better than unimodal models: 84\% accuracy compared to unimodal models hovering between 60 and 70\%
- they found that by repeating the training and testing multiple times, the distributions of the multimodal model's performance was statistically significantly better than the unimodal models
    - although they used an innapropriate statistical test for this (they used the number of repeats instead of the sample size)
- a limitation of their study is that the 3 unimodal models were trained separately and then concatenated
    - the intermediate weights and feature maps of the different modalities never interact
    - training the models altogether might allow the model to learn more complex relationships between the data and improve performance further

Extended with random walker in Meier 2020
- model was extended by adding another modality of simulated TDP-43 accumulation levels, which improved the performance


In conclusion, multimodal data fusion is a promising approach for MND prognosis, and has been shown to improve performance in some studies.
The method of data fusion is important, and concatenation is a common pitfall with small sample sizes that can be avoided with dimensionality reduction techniques and deep learning.

Next section: need to understand the predictors of survival in our dataset and see if they match up with literature

