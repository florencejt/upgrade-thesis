\chapter{Literature Review}
\label{literature_review}

Introduction to the lit review: will be looking at motor neuron disease and prognostic factors associated with it including clinical data, imaging, genetics, and fluid biomarkers.
Then we will look at machine learning techniques to predict prognosis in MND using each of these data types, and finally looking at how these data have been combined in multimodal machine learning models.

MND is multifactorial and affects multiple aspects of a patients life: motor, cognitive, behavioural.
Therefore it is also multifactorial in its prognosis: there are many factors that can affect how long a patient will live with MND, and many different ways to measure this.


\section{Clinical prognostic factors}

\subsection{Non-genetic}

\subsection{Genetic}

\subsection{Tools for Prognosis}

\section{Imaging prognostic factors}

Recap on what imaging is used for traditionally in clinical care.

\subsection{Neuroimaging}

Broken into different types of prognosis: ALSFRS-R, progression rate, survival.
Then broken into the different imaging types from there.

\subsection{Spinal Cord Imaging}

\section{Fluid Biomarkers prognostic factors}


\section{Machine Learning with clinical data}

Link to previous section: we know a lot of prognostic factors in MND, could tools be developed for clinicans to use these factors to automate or improve clinical decisions like diagnosis and prognosis?


\begin{itemize}
    \item What is machine learning?
    \item How is it different to statistics?
    \item What are the different types of machine learning?
    \begin{itemize}
        \item Supervised learning
        \item Unsupervised learning
    \end{itemize}
    \item In MND it can be done for prognosis: which is what we're looking at but also for diagnosis
    \item Honourable mention of some studies that have used machine learning for diagnosis
\end{itemize}

What are people trying to predict: ALSFRS, combined endpoint, death, clustering etc.

Challenges: PROACT and Prize4Life, IDPP CLEF

\subsection{Classical ML prognosis}

\subsection{Deep Learning prognosis}

\section{Machine Learning with imaging data}


\section{Machine Learning with multimodal data}

The papers that use multimodal data and machine learning for MND.
\begin{itemize}
    \item Neurodegenerative diseases are multifactorial and sometimes obscure
    \item Example of handedness linked to site of onset and CST marker
    \item Data fusion can help us understand them and the underlying mechanisms
    \item What about using both clinical and imaging data?
    \item This is known as multimodal data fusion methods
    \item Only done a few times in MND
\end{itemize}

Examples are thin on the ground so we're going to look at diagnosis as well as prognosis.



