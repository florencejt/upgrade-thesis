\chapter{Literature Review}
\label{literature_review}

Introduction to the lit review: will be looking at motor neuron disease and prognostic factors associated with it including clinical data, imaging, genetics, and fluid biomarkers.
Then we will look at machine learning techniques to predict prognosis in MND using each of these data types, and finally looking at how these data have been combined in multimodal machine learning models.

\section{Prognostic factors in MND}

MND is multifactorial and affects multiple aspects of a patients life: motor, cognitive, behavioural.

Therefore it is also multifactorial in its prognosis: there are many factors that can affect how long a patient will live with MND, and many different ways to measure this.


\subsection{Clinical Data}

Types of clinical data: questionnaires for motor and cognitive changes, demographics, etc

How are the clinical data used? Clinical trials, diagnosis, progression monitoring

\subsection{Imaging Data}

What is imaging data used for? Ruling out mimic diseases

\subsubsection{Neuroimaging}

Structural, DTI, fMRI

\subsubsection{Spinal Cord Imaging}


\subsubsection{Muscle Imaging}


\subsection{Genetics}


\subsection{Fluid Biomarkers}



\section{Prognosis with Machine Learning in MND}

Link to previous section: we know a lot of prognostic factors in MND, could tools be developed for clinicans to use these factors to automate or improve clinical decisions like diagnosis and prognosis?


\begin{itemize}
    \item What is machine learning?
    \item How is it different to statistics?
    \item What are the different types of machine learning?
    \begin{itemize}
        \item Supervised learning
        \item Unsupervised learning
    \end{itemize}
    \item In MND it can be done for prognosis: which is what we're looking at but also for diagnosis
    \item Honourable mention of some studies that have used machine learning for diagnosis
\end{itemize}

\subsection{With clinical data}

\begin{itemize}
    \item Classical machine learning models
    \begin{itemize}
        \item Random Forest
        \item Support Vector Machines
    \end{itemize}
    \item Deep learning models with clinical data
    \begin{itemize}
        \item Examples of studies that use deep learning with ALS
    \end{itemize}
\end{itemize}

\subsection{With imaging data}

\begin{itemize}
    \item Examples of studies that use imaging with ALS
    \item Brain MRI
    \begin{itemize}
        \item Structural MRI
        \item Diffusion MRI
    \end{itemize}
    \item Spinal cord MRI
    \item Muscle MRI/ultrasound
\end{itemize}


\subsection{With multimodal data}

The papers that use multimodal data and machine learning for MND

\begin{itemize}
    \item Neurodegenerative diseases are multifactorial and sometimes obscure
    \item Data fusion can help us understand them and the underlying mechanisms
    \item What about using both clinical and imaging data?
    \item This is known as multimodal data fusion methods
    \item Only done a few times in MND
\end{itemize}


