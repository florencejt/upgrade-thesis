\chapter{Literature Review}
\label{literature_review}

\section{Motor Neuron Disease}

\subsection{General}



\begin{itemize}
    \item What is MND?
    \item What are the different types of MND?
    \item What are the symptoms of MND?
    \item What are the causes of MND?
    \item What are the treatments for MND?
    \item What is the prognosis for MND?
    \item What is the epidemiology of MND?
\end{itemize}

\subsection{Diagnosis and Prognosis}

\begin{itemize}
    \item How is MND diagnosed?
    \begin{itemize}
        \item Diagnostic delay
    \end{itemize}
    \item How is MND prognosis predicted in clinical practice?
\end{itemize}

\subsection{Data in MND Clinical Care}

\begin{itemize}
    \item What data is collected in MND clinical care?
    \begin{itemize}
        \item Questionnaires: ALSFRSr, ALSAQ-40, etc
        \item Genetics: C9orf72, SOD1, etc - Genetic testing and counselling
        \item Imaging: MRI to rule out other conditions
        \item Fluid biomarkers: CSF and blood
        \item Other: other data types not commonly collected
    \end{itemize}
    \item Issues with data collection
    \begin{itemize}
        \item Different levels of data collection across centres
    \end{itemize}
    \item How is the data used?
    \begin{itemize}
        \item Clinical trials - inclusion/exclusion criteria
        \item Clinical care - monitoring disease progression
        \item Research - understanding disease progression
    \end{itemize}
\end{itemize}


\subsection{Data Analysis in MND}

\begin{itemize}
    \item General data analysis: survival models and hazard ratios
    \begin{itemize}
        \item Used in clinical trial reporting
    \end{itemize}
    \item Link to next section: machine learning in MND
    \begin{itemize}
        \item Took off with the PRO-ACT challenge
    \end{itemize}
\end{itemize}

\section{Machine Learning}

\subsection{Introduction to Machine Learning}

\begin{itemize}
    \item What is machine learning?
    \item How is it different to statistics?
    \item What are the different types of machine learning?
    \begin{itemize}
        \item Supervised learning
        \item Unsupervised learning
    \end{itemize}
\end{itemize}

\subsection{Machine Learning in MND}

\begin{itemize}
    \item Classical machine learning models
    \begin{itemize}
        \item Random Forest
        \item Support Vector Machines
    \end{itemize}
    \item Deep learning models with clinical data
    \begin{itemize}
        \item Examples of studies that use deep learning with ALS
    \end{itemize}
    \item Imaging in MND
    \begin{itemize}
        \item Examples of studies that use imaging with ALS
        \item Brain MRI
        \begin{itemize}
            \item Structural MRI
            \item Diffusion MRI
        \end{itemize}
        \item Spinal cord MRI
        \item Muscle MRI/ultrasound
    \end{itemize}
    \item Link to next section: multimodal data fusion
    \begin{itemize}
        \item Multimodal data fusion is a type of machine learning
    \end{itemize}
\end{itemize}

\section{Multimodal Data Fusion}

\subsection{What is Data Fusion?}

\begin{itemize}
    \item Explain what data fusion is in simple terms (with example)
    \item Can we done with and without deep learning
    \begin{itemize}
        \item We're going to focus on deep learning
    \end{itemize}
    \item Called by lots of different names: very wide reaching in literature but without lots of links between studies
    \item Used in lots of applications
    \begin{itemize}
        \item Agriculture, disaster management, etc
    \end{itemize}
    \item Link to next section: data fusion in healthcare
\end{itemize}
\subsection{Data Fusion in Healthcare}

\begin{itemize}
    \item Why is data fusion useful in healthcare?
    \begin{itemize}
        \item Multifactorial diseases require multiple data types
        \item Data is expensive to collect
        \item We want to use all the data we have
    \end{itemize}
    \item Examples of data fusion in healthcare
    \begin{itemize}
        \item Oncology
        \item Etc
    \end{itemize}
    \item Data fusion seen as useful for linking MRI with other data types
    \begin{itemize}
        \item Link to next section: data fusion in neurodegenerative disease
    \end{itemize}
\end{itemize}

\subsection{Data Fusion in Neurodegenerative Disease}

\begin{itemize}
    \item Neurodegenerative diseases are multifactorial and sometimes obscure
    \item Data fusion can help us understand them and the underlying mechanisms
    \item Examples of data fusion in neurodegenerative disease
    \begin{itemize}
        \item Alzheimer's disease: example
        \item Parkinson's disease: example
        \item Huntington's disease: example
    \end{itemize}
    \item Link to next section: if data fusion is useful in these multifactorial data fusion in MND
\end{itemize}

\subsection{Data Fusion in MND}

\begin{itemize}
    \item I've shown many different ways of doing data fusion and that it can be useful
    \item What about in MND?
    \begin{itemize}
        \item The studies that do MND data fusion
    \end{itemize}
    \item Why is my research different?
\end{itemize}