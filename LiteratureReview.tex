\chapter{Literature Review}
\label{literature_review}

This chapter contains a review of literature on prognosis of motor neuron disease, looking at both prognostic factors from various data types, and also attempts to predict prognosis using machine learning.

\section{Clinical prognostic factors}

How can prognosis be measured? Survival, rate of progression, future ALSFRS-R, future need for therapies, composite end point of NIV, tracheostomy, and death.

\subsection{Non-genetic}

A large meta-analysis in 2021 collated research studies on non-genetic prognostic factors in ALS associated with survival risk~\cite{suPredictorsSurvivalPatients2021}.
The review calculated and reported the hazard ratios (HRs) for each factor which had at least 3 studies reporting on it. Heterogeniety was assessed using the I-squared statistic, and sensitivity analysis was performed to assess the robustness of the results.
None of the sensitivity analyses showed a significant change in the results, and the authors concluded that the results were robust.

Starting with demographic factors, the meta-analysis found a higher age of symptom onset is associated with a higher risk of death (HR = 1.03)~\cite{suPredictorsSurvivalPatients2021}. However, age of onset is an imprecise measure because it is diffiuclt for patients to pinpoint their exact date of symptom onset, since symptoms often arise gradually.
It is common in clinical records for the onset date to be the first date of the month or even the first day of the year if the patient is having trouble remembering. Nevertheless, the harmful effect of older age at onset is consistent with much of literature (FIND CITATIONS HERE).
The meta-analysis also found that single, as opposed to married, patients have a worse prognosis (HR = 1.73), and that being a current smoker (HR=1.37) or a former smoker (HR=1.16) are both negative prognostic factors~\cite{suPredictorsSurvivalPatients2021}.

Rapid weight loss is a common feature of MND because of difficulties eating and swallowing, decreased appetite, and loss of muscle mass. A higher body-mass index (BMI) at diagnosis is associated with a lower risk of death (HR = 0.97) from 17 different studies in the meta-analysis ~\cite{suPredictorsSurvivalPatients2021}.
Another smaller meta-analysis specifically on BMI also found that higher BMI at baseline is protective HR=0.96)~\cite{dardiotisBodyMassIndex2018}, also backed up by a large population study of 1,809 patients from China (BMI >= 25 kg/m2, HR=0.36)~\cite{gaoEpidemiologyFactorsPredicting2021}.
However, there are some studies that have found no association between baseline BMI and survival, rather that the most important BMI prognostic factor is the rate of change in BMI~\cite{jawaidDecreaseBodyMass2010}.
Specifically, a decrease in BMI both 5 years and 10 years before symptom onset has been associated with shorter survival, in a study of N=381 ALS patients~\cite{goutmanBodyMassIndex2023}.
Since BMI is a simplistic measure that fails to capture body composition~\cite{rothmanBMIrelatedErrorsMeasurement2008}, Lindauer and colleagues used MRI of the knees and diaphragm to measure levels of subcutaneous and visceral fat in 62 ALS patients.
They found that higher subcutaneous fat is associated with higher ALSFRS-R and lower rate of ALSFRS-R decrease~\cite{lindauerAdiposeTissueDistribution2013}. Visceral fat was not associated with either measures.

The appearance of cognitive and behavioural impairment can also have an impact on survival.
Executive dysfunction, the inability to manage cognitive tasks, is associated with fast disease progression~\cite{elaminExecutiveDysfunctionNegative2011} and shorter survival in the meta-analysis (HR=2.1)~\cite{suPredictorsSurvivalPatients2021}.
Moreover, the meta-analysis found that the presence of FTD (HR=2.98) and non-specific dementia (HR=1.41) are both associated with shorter survival~\cite{suPredictorsSurvivalPatients2021}.

The El Escorial criteria are used to diagnose ALS, and a patient being assigned to the "probable" category as opposed to the "definite" category is associated with longer survival (HR=0.73), and "possible" is associated with even longer survival (HR=0.60)~\cite{suPredictorsSurvivalPatients2021}.
Definite ALS patients progressing faster also found in a large multi-centre study~\cite{westenengPrognosisPatientsAmyotrophic2018}. However, in the large Chinese cohort of 1,809 patients, there was no significant relationshop between El Escorial and survival~\cite{gaoEpidemiologyFactorsPredicting2021}.

The meta-analysis also considered the association of survival with site of onset, where motor symptoms first appear.
In comparison with spinal onset, which is the most typical ALS onset site, both respiratory onset (HR=2.2) and bulbar onset (HR=1.35), meaning the first symptoms are in the mouth and throat, are associated with shorter survival~\cite{suPredictorsSurvivalPatients2021}.
The onset sites that are protective compared to spinal onset are flail arm or leg onset (HR=0.61) and predominantly upper or lower motor neuron (pUMN or pLMN) (HR=0.32).
The speed of progression of the disease is also a prognostic factor. Fujimura-Kiyono and colleagues found that, in a cohort of 150 sporadic ALS patients, a shorter interval between the first motor onset and the next site involvement is associated with shorter survival, independent of what those onset sites are~\cite{fujimura-kiyonoOnsetSpreadingPatterns2011a}.

A higher ALSFRS-R at baseline is associated with longer survival (HR=0.96), and a higher rate of decline in ALSFRS-R from onset to diagnosis is associated with shorter survival (HR=1.48 for categorical, HR=2.37 continuous)~\cite{suPredictorsSurvivalPatients2021}.
The rate of decline in ALSFRS-R from onset to diagnosis is also called the progression rate to baseline, or PRB, and is calculated as
\begin{equation}
    PRB = \frac{48-\textnormal{ALSFRS-R}(t_{diag})}{t_{diag}-t_{onset}},
\end{equation}\label{eq:PRB}
where $t_{diag}$ and $t_{onset}$ are the dates of MND diagnosis and symptom onset respectively, and 48 is the maximum score of ALSFRS-R.

Other clinical factors associated with shorter survival are lower forced vital capacity (FVC), and a shorter time between symptom onset and diagnosis~\cite{suPredictorsSurvivalPatients2021}.
When the diagnostic delay is longer than one year, Su and colleagues found it to be associated with longer survival in their meta-analysis (HR=0.39)~\cite{suPredictorsSurvivalPatients2021}, and results from the large Chinese cohort agree that an onset shorter than one year is harmful to survival risk (HR=3.43).
This is probably because the longer the delay, the more likely it is that the patient has a slowly progressing form of MND that is not obviously ALS in diagnostic tests.

Statins, a drug used to inhibit cholesterol synthesis, has also been investigated as a prognostic factor in MND.
Su and colleagues found that taking or not taking statins has no significant effect on survival, based on 3 papers that all reported non-significant HRs~\cite{suPredictorsSurvivalPatients2021}.
However, a study published after the meta-analysis looked not only at statin use but also at the dose of statins and the duration of statin use.
Weisskopf and colleagues found that, in a cohort of 948 ALS patients, taking statins for under 3 years is protective for survival (HR=0.77), but that there is no significant protection or harm when the duration of statin use is over 3 years~\cite{weisskopfStatinMedicationsAmyotrophic2022}.
Additionally, they found that taking low-potency statins compared to not taking any statins is protective for survival (HR=0.82), but they found no signficant effects with higher-dose statins.
They concluded that the statins may have a protective effect on ALS survival, if the underlying reasons for taking statins do not necessitate high dose and long duration of use, which would imply a more severe cardiovascular condition that may be harmful to survival.

Finally, the meta-analysis found that taking Riluzole is associated with longer survival (HR=0.80)~\cite{suPredictorsSurvivalPatients2021}.

\subsection{Genetic}

Genetic testing after diagnosis is becoming more common now that the extent of the genetic contribution to MND is better understood and that therapeutics are being developed to target specific genetic mutations~\cite{efnstaskforceondiagnosisandmanagementofamyotrophiclateralsclerosis:EFNSGuidelinesClinical2012}.
Increased data availability of genetic markers has led to a number of studies on the prognostic value of genetic markers in MND.
A network meta-analysis on genetic factors associated with survival in ALS found that the C9orf72 repeat expansion is associated with shorter survival compared to no known variants (HR=1.6)~\cite{suGeneticFactorsSurvival2022}.
This is consistent with another study of cohort of 1,245 ALS patients that found that, in a multivariable Cox regression model, the C9orf72 repeat expansion is associated with shorter survival (HR=1.65)~\cite{chioAssociationCopresencePathogenic2023}.
Also associated with shorter survival is ATXN2, a CAG repeat expansion usually associated with spinal onset ALS (HR=3.6), and a mutated FUS (fused in sarcoma) (HR=1.8)~\cite{suGeneticFactorsSurvival2022}.

\subsection{Tools for Prognosis}

There are limited clinical tools for prognosis prediction in MND. The most common way for progression to be assessed is through the progression rate in Equation~\ref{eq:PRB}, which can be generalised to the progression rate to any time point, not just diagnosis.
Extrapolating this progression rate to predict future ALSFRS-R scores is a common way to predict future progression, called the "pre-slope model", but it is not a perfect method since it assumes that ALSFRS-R decline is linear.

To overcome these issues, the D50 model was developed, which fits a sigmoidal curve to a patient's ALSFRS-R timepoints, yielding an individualised prediction of future ALSFRS-R scores
~\cite{poesenNeurofilamentMarkersALS2017, steinbachApplyingD50Disease2020}.
The D50 model can provide a measure of disease aggressiveness, which is the estimated rate of functional loss from the sigmoidal curve, and disease accummulation, which is the patient's position on the D50 curve independent of time.

A non-linear extension of the D50 model was proposed by Ramamoorthy and colleagues, where Gaussian processes are used to non-parametrically cluster patients into non-linear ALSFRS-R trajectories~\cite{ramamoorthyIdentifyingPatternsAmyotrophic2022}.
In their model development, they found that many of the patients in their cohort had non-linear ALSFRS-R trajectories (convex, concave, sigmoidal) and that the slope of ALSFRS-R in the first year of the disease is not sufficient to accurately predict future ALSFRS-R scores.

Finally, a popular model for predicting survival probability in ALS is the ENCALS model~\cite{westenengPrognosisPatientsAmyotrophic2018}.
Using data from 14 European ALS centres and over 11,000 patient records, Westeneng and colleagues developed a multivariable Royston-Parmar model to predict a survival probability function for an individual ALS patient from baseline information.
They selected the predictors through backward elimination and bootstapping, and predictors present in more than 70\% of the bootstrapped models were included in the final model.
The harmful predictors included in the final model are bulbar onset (HR=1.71), age of onset (HR=1.03), El Escorial definite ALS (HR=1.47), higher PRB (HR=6.33), presence of FTD (HR=1.34), and C9orf72 repeat expansion (HR=1.45).
The protective predictors are a longer diagnostic delay (HR=0.52) and a higher FVC (HR=0.99).

The ENCALS model is available by request of medical doctors, and was validated using a leave-one-site-out protocol.
The authors demonstrated the model's ability to predict survival by using data from Stephen Hawking, a famous physicist who lived with ALS for a startling 55 years after diagnosis, and found that the model predicted his survival probability well.
The predicted probability of surviving over 10 years was 94\% and the interquartile range of predicted year of death was 1981 to 2011, which is consistent with the year of Stephen Hawking's tracheostomy in 1985 (which is the endpoint of the model's prediction: NIV for more than 23 hours a day, tracheostomy, or death).




\section{Imaging prognostic factors}

Recap on what imaging is used for traditionally in clinical care.

\subsection{Neuroimaging}

Broken into different types of prognosis: ALSFRS-R, progression rate, survival.
Then broken into the different imaging types from there.

\subsection{Spinal Cord Imaging}

\section{Fluid Biomarkers prognostic factors}


\section{Machine Learning with clinical data}

Link to previous section: we know a lot of prognostic factors in MND, could tools be developed for clinicans to use these factors to automate or improve clinical decisions like diagnosis and prognosis?


\begin{itemize}
    \item What is machine learning?
    \item How is it different to statistics?
    \item What are the different types of machine learning?
    \begin{itemize}
        \item Supervised learning
        \item Unsupervised learning
    \end{itemize}
    \item In MND it can be done for prognosis: which is what we're looking at but also for diagnosis
    \item Honourable mention of some studies that have used machine learning for diagnosis
\end{itemize}

What are people trying to predict: ALSFRS, combined endpoint, death, clustering etc.

Challenges: PROACT and Prize4Life, IDPP CLEF

\subsection{Classical ML prognosis}

\subsection{Deep Learning prognosis}

\section{Machine Learning with imaging data}


\section{Machine Learning with multimodal data}

The papers that use multimodal data and machine learning for MND.
\begin{itemize}
    \item Neurodegenerative diseases are multifactorial and sometimes obscure
    \item Example of handedness linked to site of onset and CST marker
    \item Data fusion can help us understand them and the underlying mechanisms
    \item What about using both clinical and imaging data?
    \item This is known as multimodal data fusion methods
    \item Only done a few times in MND
\end{itemize}

Examples are thin on the ground so we're going to look at diagnosis as well as prognosis.



