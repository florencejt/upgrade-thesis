\chapter{Literature Review}
\label{literature_review}

This chapter contains a review of literature on prognosis of motor neuron disease, looking at both prognostic factors from various data types, and also attempts to predict prognosis using machine learning.

\section{Clinical prognostic factors}

How can prognosis be measured? Survival, rate of progression, future ALSFRS-R, future need for therapies.

\subsection{Non-genetic}

A large meta-analysis in 2021 collated research studies on non-genetic prognostic factors in ALS associated with survival risk~\cite{suPredictorsSurvivalPatients2021}.
The review calculated and reported the hazard ratios (HRs) for each factor which had at least 3 studies reporting on it. Heterogeniety was assessed using the I-squared statistic, and sensitivity analysis was performed to assess the robustness of the results.
None of the sensitivity analyses showed a significant change in the results, and the authors concluded that the results were robust.

Starting with demographic factors, the meta-analysis found a higher age of symptom onset is associated with a higher risk of death (HR = 1.03)~\cite{suPredictorsSurvivalPatients2021}. However, age of onset is an imprecise measure because it is diffiuclt for patients to pinpoint their exact date of symptom onset, since symptoms often arise gradually.
It is common in clinical records for the onset date to be the first date of the month or even the first day of the year if the patient is having trouble remembering. Nevertheless, the harmful effect of older age at onset is consistent with much of literature (FIND CITATIONS HERE).
The meta-analysis also found that single, as opposed to married, patients have a worse prognosis (HR = 1.73), and that being a current smoker (HR=1.37) or a former smoker (HR=1.16) are both negative prognostic factors~\cite{suPredictorsSurvivalPatients2021}.

Rapid weight loss is a common feature of MND because of difficulties eating and swallowing, decreased appetite, and loss of muscle mass. A higher body-mass index (BMI) at diagnosis is associated with a lower risk of death (HR = 0.97) from 17 different studies in the meta-analysis ~\cite{suPredictorsSurvivalPatients2021}.
Another smaller meta-analysis specifically on BMI also found that higher BMI at baseline is protective HR=0.96)~\cite{dardiotisBodyMassIndex2018}, also backed up by a large population study of 1,809 patients from China (BMI >= 25 kg/m2, HR=0.36)~\cite{gaoEpidemiologyFactorsPredicting2021}.
However, there are some studies that have found no association between baseline BMI and survival, rather that the most important BMI prognostic factor is the rate of change in BMI~\cite{jawaidDecreaseBodyMass2010}.
Specifically, a decrease in BMI both 5 years and 10 years before symptom onset has been associated with shorter survival, in a study of N=381 ALS patients~\cite{goutmanBodyMassIndex2023}.
Since BMI is a simplistic measure that fails to capture body composition~\cite{rothmanBMIrelatedErrorsMeasurement2008}, Lindauer and colleagues used MRI of the knees and diaphragm to measure levels of subcutaneous and visceral fat in 62 ALS patients.
They found that higher subcutaneous fat is associated with higher ALSFRS-R and lower rate of ALSFRS-R decrease~\cite{lindauerAdiposeTissueDistribution2013}. Visceral fat was not associated with either measures.

The appearance of cognitive and behavioural impairment can also have an impact on survival.
Executive dysfunction, the inability to manage cognitive tasks, is associated with fast disease progression~\cite{elaminExecutiveDysfunctionNegative2011} and shorter survival in the meta-analysis (HR=2.1)~\cite{suPredictorsSurvivalPatients2021}.
Moreover, the meta-analysis found that the presence of FTD (HR=2.98) and non-specific dementia (HR=1.41) are both associated with shorter survival~\cite{suPredictorsSurvivalPatients2021}.

\subsection{Genetic}

\subsection{Tools for Prognosis}

\section{Imaging prognostic factors}

Recap on what imaging is used for traditionally in clinical care.

\subsection{Neuroimaging}

Broken into different types of prognosis: ALSFRS-R, progression rate, survival.
Then broken into the different imaging types from there.

\subsection{Spinal Cord Imaging}

\section{Fluid Biomarkers prognostic factors}


\section{Machine Learning with clinical data}

Link to previous section: we know a lot of prognostic factors in MND, could tools be developed for clinicans to use these factors to automate or improve clinical decisions like diagnosis and prognosis?


\begin{itemize}
    \item What is machine learning?
    \item How is it different to statistics?
    \item What are the different types of machine learning?
    \begin{itemize}
        \item Supervised learning
        \item Unsupervised learning
    \end{itemize}
    \item In MND it can be done for prognosis: which is what we're looking at but also for diagnosis
    \item Honourable mention of some studies that have used machine learning for diagnosis
\end{itemize}

What are people trying to predict: ALSFRS, combined endpoint, death, clustering etc.

Challenges: PROACT and Prize4Life, IDPP CLEF

\subsection{Classical ML prognosis}

\subsection{Deep Learning prognosis}

\section{Machine Learning with imaging data}


\section{Machine Learning with multimodal data}

The papers that use multimodal data and machine learning for MND.
\begin{itemize}
    \item Neurodegenerative diseases are multifactorial and sometimes obscure
    \item Example of handedness linked to site of onset and CST marker
    \item Data fusion can help us understand them and the underlying mechanisms
    \item What about using both clinical and imaging data?
    \item This is known as multimodal data fusion methods
    \item Only done a few times in MND
\end{itemize}

Examples are thin on the ground so we're going to look at diagnosis as well as prognosis.



